%  Synergies and mergers-Thursday, 19 November 2020
%
%  Created by Patrick on 2020-11-19.
%  Copyright (c) 2020 __Patrick Legros__. All rights reserved.
%
%!TEX encoding = UTF-8 Unicode  

\documentclass[a4paper]{article}
\usepackage[utf8]{inputenc}
\usepackage{amsmath}
\usepackage{amssymb}
\usepackage{url}
\usepackage{hyperref}
%\usepackage[round]{natbib}
%\bibliographystyle{plainnat}
\usepackage[mathscr]{euscript}
\let\euscr\mathscr \let\mathscr\relax
\usepackage[scr]{rsfso}
\usepackage{graphicx}
\usepackage{float}
\usepackage{enumerate}
\usepackage{blindtext}
\usepackage{setspace}
\usepackage{eurosym}
\usepackage{hyperref}
\usepackage{pdflscape}
\usepackage{array, multirow}
\newcommand*{\qed}{\hfill\ensuremath{\blacksquare}}%
\usepackage[dvipsnames]{xcolor}

\usepackage{todonotes}

\usepackage{tikz}
\hypersetup{
colorlinks,
citecolor=NavyBlue,
linkcolor=NavyBlue,
urlcolor=NavyBlue}
\usetikzlibrary{matrix}
\newtheorem{prop}{Proposition}
\newtheorem{lemma}{Lemma}
\newtheorem{corollary}{Corollary}
\newtheorem{ass}{Assumption}
\newtheorem{result}{Result}
\newtheorem{condition}{Condition}
\newtheorem{Theorem}{Theorem}
\newtheorem{Definition}{Definition}

\newenvironment{reusefigure}[2][htbp]
  {\addtocounter{figure}{-1}%
   \renewcommand{\theHfigure}{dupe-fig}% If you're using hyperref
   \renewcommand{\thefigure}{\ref{#2}}% Figure counter is \ref
   \renewcommand{\addcontentsline}[3]{}% Avoid placing figure in LoF
   \begin{figure}[#1]}
  {\end{figure}}
  
  \usepackage{titlesec}

%\setcounter{secnumdepth}{4}

\titleformat{\paragraph}
{\normalfont\normalsize\bfseries}{\theparagraph}{1em}{}
\titlespacing*{\paragraph}
{0pt}{3.25ex plus 1ex minus .2ex}{1.5ex plus .2ex}

 \usepackage{geometry}

 
 \newcommand{\E}{\mathbb E}
 \newcommand{\M}{\mathcal M}
\renewcommand{\th}{\hat\theta}
\renewcommand{\t}{\theta}

\begin{document}


\title{Synergistic information$^{3.0}$}
\author{Antoine Dubus and Patrick Legros\thanks{We would like to thank.}}
\date{\today}

%\author{Antoine Dubus\thanks{~~Université Libre de Bruxelles, ECARES; \href{mailto:antoine1dubus@gmail.com}{antoine1dubus@gmail.com}.}}

\maketitle

\begin{abstract}

\noindent 

\end{abstract}
 
\textbf{Very preliminary, do not circulate.}

\baselineskip0.7cm

\section{Literature}
\begin{itemize}\setlength\itemsep{-1em}
    \item Optimal merger policy when firms have private information \cite{Besanko1993}
    \item Information sharing in oligopolies
    \item Data as assets
    \item Computer science and data synergies
    \item Joint ventures before merger
\end{itemize}

\section{Model}



\bibliographystyle{plain}
\bibliography{Bibliography}

$l>0$, if share $s$ and compete profits are $X-\t l s$ for $1$ and $\t s$ for $2$. Firm $2$ has full negotiation power. There are two relevant levels of sharing: $s=0$ and $s^*$. If $0$, firm $2$ does not know $\t$ at the time of merger; if sharing is $s^*$, firm $2$ knows $\t$.

$\t$ has ex-ante distribution $F(\t)$, with continuous density, MLRP satisfied.

Note: given $s^*$, the threat point of firm $1$ is a decreasing function of $\t$ while the threat point of firm $2$ is an increasing function of $\t$. Since firm $2$ has full bargaining power, firm $1$ accepts a price for her asset of $p(\t)=X-\t l s^*$, and firm $2$ has a surplus from merger of $X(1+\t)-p(\t)=\t (X-(1-l)s^*$. Ex-ante, the parties have to agree on a price $T$ for sharing $s^*$. Type $\t$ accepts to share if $p(\t)+T\geq u_0(\t)$, where $u_0(\t)$ is $\t$'s payoff if there is no sharing of information.

(Note that if firm $2$ offers a price $T$ for a sharing of $s^*$, and a set $\Theta(T)$ refuses the offer, when the merger possibility arises, firm $2$ believes that firm $1$ is of type in $\Theta(T)$. 

In a revelation game at the merging stage, the variation of equilibrium utility given to type $\t$ is (assuming that $\Theta(T)$ has compact support, hence that the conditional distribution has a positive density on its support) is increasing in $\t$ (note that all type in $\Theta(t))$ have the same outside option of $X$).

The value to the merger is $\frac{1}{2}X(1+\t)$. There is no loss of generality in assuming that firm $1$ gets all the surplus in exchange for paying a price to firm $2$. 

A mechanism is $(p(\t),z(\t))$, where $p(\t)$ is the price paid by firm $1$ to firm $2$ and $z(\t)$ is the probability that firm $1$ agrees to the merger. The participation constraint is
\begin{equation}
  -p(\t)+z(\t)X(1+\t)+(1-z(\t))X\geq X 
\end{equation}
or
\begin{equation*}
  p(\t)\leq z(\t)X \t
\end{equation*}
while the truth-telling constraint is
\begin{equation*}
  \t \in \arg\max U(\th|\t):=-p(\th)+X+ z(\th)X\t
\end{equation*}
%
Usual manipulations yield 

%
\[
  X z(\th)(\t-\th)\leq U(\t)-U(\th)\leq Xz(\t)(\t-\th)
\]
%
hence that $U(\t)$ and $z(\t)$ are non-decreasing function. Moreover, by the envelop theorem, $\dot U(\theta)=z(\t)X$. 

Hence, firm $2$ offers a mechanism $(p,z)$ to solve
\begin{align*}
  \max_{(p(\cdot);z(\dot))} \int \left(-U(\t)+X+Xz(\t)\t \right)&dF(\t)\\ 
  U(\t) \geq X & \hspace{2cm} \text{(IR)} \\ 
  \dot{U}(\t)=z(\t)X & \hspace{2cm} \text{(IC)} \\ 
\end{align*}
Standard derivations show that $U(\underline \t)=X$ and that $z(\t)$ solves  

\begin{align*}
\max_{\{z(\t)\}} &\int_{\underline \t}^{\overline \t}z(\t)\left( \t-\frac{1-F(\t)}{f(\t)}\right)f(\t)d\t.
\end{align*}
%
Clearly, by MLRP, there exists $t^*$ such that the optimal solution is to set $z(\t)=0$ for $\t<\t^*$ and $z(\t)=1$ for $\t>\t^*$. 
%






\end{document}

