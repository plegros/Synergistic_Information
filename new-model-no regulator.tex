%  Synergies and mergers-Thursday, 19 November 2020
%
%  Created by Patrick on 2020-11-19.
%  Copyright (c) 2020 __Patrick Legros__. All rights reserved.
%
%!TEX encoding = UTF-8 Unicode  

\documentclass[a4paper]{article}
\usepackage[utf8]{inputenc}
\usepackage{amsthm,amsmath,amssymb,amsfonts}
\usepackage{url}
\usepackage{hyperref}
\usepackage[round]{natbib}
%\bibliographystyle{plainnat}
\usepackage[mathscr]{euscript}
\let\euscr\mathscr \let\mathscr\relax
\usepackage[scr]{rsfso}
\usepackage{graphicx}
\usepackage{float}
\usepackage{enumerate}
\usepackage{blindtext}
\usepackage{setspace}
\usepackage{eurosym}
\usepackage{hyperref}
\usepackage{pdflscape}
\usepackage{array, multirow}
\usepackage[dvipsnames]{xcolor}

\usepackage{todonotes}

\usepackage{tikz}
\hypersetup{
colorlinks,
citecolor=NavyBlue,
linkcolor=NavyBlue,
urlcolor=NavyBlue}
\usetikzlibrary{matrix}
\newtheorem{prop}{Proposition}
\newtheorem{lemma}{Lemma}
\newtheorem{corollary}{Corollary}
\newtheorem{ass}{Assumption}
\newtheorem{result}{Result}
\newtheorem{condition}{Condition}
\newtheorem{Theorem}{Theorem}
\newtheorem{Definition}{Definition}
\newtheorem{remark}{Remark}

\newenvironment{reusefigure}[2][htbp]
  {\addtocounter{figure}{-1}%
   \renewcommand{\theHfigure}{dupe-fig}% If you're using hyperref
   \renewcommand{\thefigure}{\ref{#2}}% Figure counter is \ref
   \renewcommand{\addcontentsline}[3]{}% Avoid placing figure in LoF
   \begin{figure}[#1]}
  {\end{figure}}
  
  \usepackage{titlesec}

%\setcounter{secnumdepth}{4}

\titleformat{\paragraph}
{\normalfont\normalsize\bfseries}{\theparagraph}{1em}{}
\titlespacing*{\paragraph}
{0pt}{3.25ex plus 1ex minus .2ex}{1.5ex plus .2ex}

 \usepackage{geometry}

 
 \newcommand{\E}{\mathbb E}
 \newcommand{\N}{\mathcal N}
\renewcommand{\th}{\hat\theta}
\renewcommand{\t}{\theta}

\begin{document}


\title{Synergistic Information$^{3.0}$}
\author{Antoine Dubus and Patrick Legros\thanks{We would like to thank.}}
\date{\today}

%\author{Antoine Dubus\thanks{~~Université Libre de Bruxelles, ECARES; \href{mailto:antoine1dubus@gmail.com}{antoine1dubus@gmail.com}.}}

\maketitle

\begin{abstract}

\noindent 

\end{abstract}
 
\textbf{Very preliminary, do not circulate.}

\baselineskip0.7cm

\section{Literature}
\begin{itemize}\setlength\itemsep{-1em}
    \item Optimal merger policy when firms have private information \cite{Besanko1993}
    \item Information sharing in oligopolies
    \item Data as assets
    \item Computer science and data synergies
    \item Joint ventures before merger
    \item incomplete contracts and cooperative investments \citep{Che1999}
\end{itemize}




\section{Introduction}
Idea: firm $1$ with dataset may have a merger opportunity in the future with another firm $2$ using also data. If there are synergies, the merger is beneficial, but without prior infomration, firm $2$ may be reluctant to merge. Revelation of information about synergies can be done in one of two ways:
\begin{itemize}
  \item By prior sharing of some of the data from firm $1$. If the amount of data that is shared is sufficiently large, this will enable firm $2$ to learn about the synergy level; otherwise there is no learning. For the example below, the assumption is that sharing below $s^*$ does not bring information, but sharing about $s^*$ brings information. A more `continuous' model will have a change in the precision of information continuous as a function of $s$.
  \item Or by waiting and bargianing under incomplete information on the part of firm $2$ about the level of the synergy. To induce revelation of information, the firms may have to walk away from the merger with some probability.
\end{itemize}
Hence, waiting to merge generates ex-post inefficiencies while not waiting insures ex-post efficiency (since there is symmetric information at the time of the merger). From firm $1$'s point of view, sharing of data brings a competitive disadvantage if the merger fails and firms compete (the case of sharing for collusion to be analyzed next). Hence, to induce firm $1$ to share, firm $2$ has to offer a high price for the shared data, and give rents to firms when the synergy is low. By contrast, waiting and bargaining under asymmetric information gives rents to firms $1$ when the synergy is high.

\section{Model}
\begin{itemize}
  \item If firm $1$ shares $s$ and the firms compete, the profits are $X-\t l s$ for $1$ and $\t s$ for $2$.
  \item Firm $2$ has full negotiation power. (Needed: firm $2$ cannot commit to a mechanism at the merging stage? Result below seems fine even if firm $2$ can commit. TO BE VERIFIED.) 
  \item There are two relevant levels of sharing: $s=0$ and $s^*$. If $0$, firm $2$ does not know $\t$ at the time of merger; if sharing is $s^*$, firm $2$ knows $\t$.
  \item $\t$ has ex-ante distribution $F(\t)$, with continuous density and support $[0,\overline \t]$ ($\overline \t=\infty$ is allowed), MLRP satisfied.
\end{itemize}


\paragraph{Steps to follow:}
\begin{enumerate}[(1)]\setlength\itemsep{0em}
  \item Let $T$ the price that firm $2$ offers to pay for data $s^*$.
  \item If firm $1$ accepts, $s^*$ is shared and firm $2$ learns $\t$. When merger opportunity arises, firm $2$ extracts all the surplus from merger. Let $U_1(\t|s=s^*)$ be the expected utility of type $\t$ of doing data sharing.
  \item Let $\N$ be the subset of types that accept $T$.
  \item\label{stage-merger} Given $\N$, there is an optimal revelation mechanism that maximizes firm $2$'s expected payoff subject to the IR and IC conditions for firm $1$. Let $U(\t|s=0)$ be the expected payoff of firm $1$ of type $\t$ when playing this mechanism (corresponding to the `sunk' belief of firm $2$ of facing firms $1$ with types in $\N$. Note that we can compute this value for any $\t$, which is obviously necessary in order to verify the incentives of different types to do data sharing or not.)
  \item Go back and verify that for each $\t\in \N$, $U_1(\t|s=0)\geq U_1(\t|s=s^*)$ and for each $\t\notin \N$, $U_1(\t|s=0)\leq U_1(\t|s=s^*)$
\end{enumerate}
%
Note that we assume that firm $2$ does not commit to the mechanism used at stage \eqref{stage-merger}. Probably not necessary, but facilitates derivations; also quite relevant in the \cite{anton2002sale} type of environment.

As an illustration of the mechanics of the model, suppose firm $2$ has full bargaining power. Firms that accept to share $s^*$ for a price of $T$ anticipate that firm $2$ will make a TIOLI offer if a merger possibily arises and will extract all the surplus. Hence, the payoff of firms that accept the offer to share is   
%
\[
U_1(\t|s=s^*)=X+T-\t l s^*)
\]
%

which is a decreasing function of $\t$. By contrast, firms that do not share data anticipate that firm $2$ will make an offer that gives them an informational rent, that is $U_1(\t|s=0)$ is increasing in $\t$. As we will see, this is a standard screening problem and, because of a lack of commitment of firm $2$, firm $1$ anticipates that the participation constraint of the lowest type $\t_0$ who does not share data will be binding and that all types greater than $\t_0$ get a rent. It follows that $\N$ is an interval $[\theta_0,\overline \t]$. Furthermore, because $\t_0$ must be indifferent between sharing and not sharing data, we need $T+X-\t_0 l s^*=X$, or
\begin{equation}\label{eq:T}
  T=\t_0 l s^*.
\end{equation}
%
\paragraph{Mechanism if firm $2$ believes that $\t\in\N$ at the merging phase.} At the merging stage, firm $2$ believes that types have a distribution with support on $\N=[\t_0,\overline \t]$. Note that all types in $\N$ do not share data and have the same outside option of $X$. The value to the merger is $X(1+\t)$. There is no loss of generality in assuming that firm $1$ gets all the surplus in exchange for paying a price to firm $2$.

A mechanism is then a menu $\{(p(\t),z(\t));t\in \N\}$, where $p(\t)$ is the price paid by firm $1$ to firm $2$ and $z(\t)$ is the probability that firm $2$ agrees to the merger. It should ve clear that if $\t_0$ does not get a rent in the mechanism, types $t<\t_0$ get a negative rent if they do not share data; by conttrast they get a positive rent equal to $ls^* (\t_0-\t)$ if they share data. The participation constraint of firm $1$ is
\begin{equation}
  U(\t):=-p(\t)+z(\t)X(1+\t)+(1-z(\t))X\geq X 
\end{equation}
or
\begin{equation*}
  p(\t)\leq z(\t)X \t
\end{equation*}
while the truth-telling constraint is
\begin{equation*}
  \t \in \arg\max_{\th} U(\th|\t):=-p(\th)+X+ z(\th)X\t
\end{equation*}
%
Usual manipulations yield 
%
\[
  X z(\th)(\t-\th)\leq U(\t)-U(\th)\leq Xz(\t)(\t-\th)
\]
%
hence that $U(\t)$ and $z(\t)$ are almost everywhere non-decreasing function. Moreover, by the envelop theorem, $\dot U(\theta)=z(\t)X$. 

Hence, firm $2$ offers a mechanism $(p,z)$ to solve $\max_{\{p(\cdot),s(\cdot)\}}\int_{\t_0}^{\overline t}p(\t)f(\t)dt$ subject to the IR and IC constraints. By using $p(\t)=X+z(\t)X\t-U(\t)$, the problem can be rewritten as
\begin{align*}
  \max_{(p(\cdot);z(\dot))} \int_{\t_0}^{\overline \t} \left(-U(\t)+X+Xz(\t)\t \right)&\frac{f(\t)}{1-F(\t_0)}d\t\\ 
  U(\t) \geq X & \hspace{2cm} \text{(IR)} \\ 
  \dot{U}(\t)=z(\t)X & \hspace{2cm} \text{(IC)} \\ 
\end{align*}
%
Standard derivations yield to the equivalent problem 
%
\begin{align*}
\max_{\{z(\t)\}} &\int_{\t\in \N}z(\t)\left( \t-\frac{1-F(\t)}{f(\t)}\right)f(\t)d\t.
\end{align*}
%
Clearly, by MLRP, there exists a unique value $\t^*$ solving
%
\[
\t^*=\frac{1-F(\t^*)}{f(\t^*)}.
\]
%
Hence, the optimal solution is to set $z(\t)=0$ for $\t<\t^*$ and $z(\t)=1$ for $\t>\t^*$. It follows that the expected payoff to firm $2$ is (using $T=\t_0 l s^*$)\footnote{
  Indeed, if $\t_0<\t^*$, firm $2$ optimally commits not to merge when $\theta\in (\t_0,\t^*)$, hence gets a surplus only if $t< \t_0$ or $t\geq \t^*$.
}
%
\begin{equation}\label{eq:V-t0}
  V(\t_0):=\begin{cases}
    \int_0^{\t_0} (X(1+\t)-\t_0 l s^*)f(\t)d\t +X \int_{\t^*}^{\overline \t}\left(\t-\frac{1-F(\t)}{f(\t)}\right)f(\t)d\t & \text{ if } \t_0\leq\t^*\\ 
    \int_0^{\t_0} (X(1+\t)-\t_0 l s^*)f(\t)d\t +X \int_{\t_0}^{\overline \t}\left(\t-\frac{1-F(\t)}{f(\t)}\right)f(\t)d\t & \text{ if } \t_0\geq\t^*
  \end{cases}
\end{equation}

%
The solution to $\max_{\t_0} V(\t_0)$ necessarily involves $\t_0>0$ under the assumption that $f(0)$ is positive, that is in equilibrium, data sharing happens.
\begin{prop}
  \begin{enumerate}[(i)]\setlength\itemsep{0em}
    \item If $f(0)$ is positive, firm $2$ chooses $\t_0>0$, that is there is data sharing in equilibrium. 
    \item If the competitive loss from data sharing $l s^*$ is greater than $X\frac{f(\overline \t)}{\overline \t(1+\t f(\overline \t))}$, in equilibrium $\t_0\in (0,1)$ and some firms choose data sharing while others, those with high type, choose not to share data.
    \item If for each $t_0>\t^*$, $ls^*>X\frac{(1+\t^*)f(\t_0)}{1+(\t_0-\t^*)f(\t_0)}$, then in equilibrium $\t<\t_0$, that is firms will not merge for $\t\in (\t_0,\t^*)$.
  \end{enumerate}
\end{prop}
\begin{proof}
  To show (i), it is enough to show that $V'(0)$ is positive. If $\t_0=0$, all firms are expected not to share data. However, as $\theta_0$ increases beyond $0$, the value of $V(\t_0)$ is given by the first function in \eqref{eq:V-t0}. Therefore, $V'(0)=Xf(0)$ is non-negative.

  To show (ii)), note that at $\t_0=\overline \t$, $\overline \t>\t^*$, and therefore in a neighborhood of $\overline \t$, $V(\t_0)$ is the function in the second part of \eqref{eq:V-t0}, where all types choose to share data. Then, 
    %
  \[
    \lim_{\t_0\uparrow \overline \t}V'(\t_0)=Xf(\overline \t)-\overline \t ls^*(1+ f(\overline \t))
 \]
 \todo[inline]{Patrick je pense qu'il manque un terme dans l'égalité, et que ca devrait etre $Xf(\overline \t)-ls^* \overline \t \frac{f(\overline \t)}{1+\overline \t}$}
  %
and under the condition in the proposition, $\lim_{\t_0\uparrow \overline \t}V'(\t_0)<0$, implying that $\t_0<\overline \t$.

To show (iii), assume that $\t_0>\t^*$. Then using the second function in \eqref{eq:V-t0}, we have
\begin{align*}
  V'(\t_0)&=(X(1+\t_0)-\t_0ls^*)f(\t_0)-F(\t_0)l s^*- X\left(\t_0-\frac{1-F(\t_0)}{f(\t_0)}\right)f(\t_0)\\ 
  &<(X(1+\t_0)-\t_0ls^*)f(\t_0)- (1-\t^* f(\t_0)) l s^*- X\left(\t_0-\t^*\right)f(\t_0)\\ 
  &=X(1+\t^*)f(\t_0)-ls^*(1+(\t_0-\t^*)f(\t_0)),
\end{align*}
  % 
  where the inequality is due to a decreasing likelihood ratio $\frac{1-F(\t)}{f(\t)}$, that implies that for $\t_0>\t^*$,  $\frac{1-F(\t_0)}{f(\t_0)}<\frac{1-F(\t^*)}{f(\t^*)}=\t^*$, hence $F(\t_0)>1-\t^*f(\t_0)$. The condition in (iii) insures that $V'(\t_0)$ is negative on $[\t^*,\overline \t]$
\end{proof}
%
An example is for the exponential distribution with positive parameter $\lambda$. For this distribution, $\t^*=\frac{1}{\lambda}$, $f(0)=\lambda$ and $f(\infty)=0$. The condition in (i) is satisfied; because $f(\infty)=0$, the condition in (ii) holds for all values of $ls^*$. Finally, the right hand side in condition in (iii) is equal to $\frac{(1+\lambda) e^{-\lambda \t_0}}{1+(\lambda \t_0-1)e^{-\lambda \t_0}}$, a decreasing function of $\t_0\geq \frac{1}{\lambda}$, hence with a maximum value of $\frac{\lambda+1}{e}$. Hence, it is sufficient for (iii) that (in fact could derive a necessary and sufficient condition by directly looking at $V'(\t)$ for the exponential distribution).
%
\[
ls^*>\frac{\lambda+1}{e}.
\]
%
%
Under condition (iii), there is no merger for types in the interval $(\t_0,\t^*)$, which is an inefficient outcome from the point of view of the industry. A conjecture is that when the bargaining power is more evenly split among the two firms, the likelihood of such an outcome decreases.

\begin{remark}
  The size of the competitive loss $ls^*$ is key.. For instance, if $l=0$, both conditions (ii)-(ii) do not hold. In this case, firm $2$ may as well offer any price for data sharing (because firm $1$ does not bear a competitive penalty). If $l$ is very large, conditions (i)-(iii) hold at $\t_0$ close to zero, and the inefficiency is maximum as there is no merger for all $\t\in[0,\t^*)$. Interpret this in light of potential data sharing between Biotechs and Big pharmas.

  \underline{We need to provide interpretations}, real life examples that could fit with the theoretical results. 
  \end{remark}

\section{EXTENSIONS TO BE DONE}  
(order not necessarily sequential)
    \begin{enumerate}[TBD 1.]\setlength\itemsep{0em}
      \item Look at the case where there is a competitive \emph{gain} of data sharing, that is $l<0$. Firm $1$ is less reluctant to share with firm $2$, so...?
      \item Look at case where sharing $s$ allows firm $2$ to learn the true value of $\t$ with probability $\alpha(s)$, an increasing function of $s$. (The case above coincides with $\alpha(s)=0$ is $s<s^*$ and $\alpha(s)=1$ if $s\geq s^*$.)
      \item The general case where firm $2$ has bargaining power $\beta<1$. Hence it is as if the two firms agree on a price $T$ and a mechanism that satisfies IR and IC for firm $1$ in order to maximize the weighted sum $(1_\beta)U_1+\beta U_2$, that is $U_1+U_2-\beta U_1$: as $\beta$ decreases. Is it less likely that the firms will \emph{not} merge?
      \item Uncertain merger opportunities. For instance a biotech may share data with a pharma who decides later on to merge with another firm or not to pursue the relationship. Or the regulator's decision is somewhat random.
      \item Endogenous merger choice by the regulator. Seems complicated but may be worth thinking about it as the paper should probably say something about guidelines for regulating data sharing.
    \end{enumerate}

\bibliographystyle{agsm}
\bibliography{biblio-synergies.bib}


\end{document}

