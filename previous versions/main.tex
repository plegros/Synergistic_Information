%  Synergies and mergers-Thursday, 19 November 2020
%
%  Created by Patrick on 2020-11-19.
%  Copyright (c) 2020 __Patrick Legros__. All rights reserved.
%
%!TEX encoding = UTF-8 Unicode  

\documentclass[a4paper]{article}
\usepackage[utf8]{inputenc}
\usepackage{amsmath}
\usepackage{amssymb}
\usepackage{url}
\usepackage{hyperref}
%\usepackage[round]{natbib}
%\bibliographystyle{plainnat}
\usepackage[mathscr]{euscript}
\let\euscr\mathscr \let\mathscr\relax
\usepackage[scr]{rsfso}
\usepackage{graphicx}
\usepackage{float}
\usepackage{enumerate}
\usepackage{blindtext}
\usepackage{setspace}
\usepackage{eurosym}
\usepackage{hyperref}
\usepackage{pdflscape}
\usepackage{array, multirow}
\newcommand*{\qed}{\hfill\ensuremath{\blacksquare}}%
\usepackage[dvipsnames]{xcolor}

\usepackage{todonotes}

\usepackage{tikz}
\hypersetup{
colorlinks,
citecolor=NavyBlue,
linkcolor=NavyBlue,
urlcolor=NavyBlue}
\usetikzlibrary{matrix}
\newtheorem{prop}{Proposition}
\newtheorem{lemma}{Lemma}
\newtheorem{corollary}{Corollary}
\newtheorem{ass}{Assumption}
\newtheorem{result}{Result}
\newtheorem{condition}{Condition}
\newtheorem{Theorem}{Theorem}
\newtheorem{Definition}{Definition}

\newenvironment{reusefigure}[2][htbp]
  {\addtocounter{figure}{-1}%
   \renewcommand{\theHfigure}{dupe-fig}% If you're using hyperref
   \renewcommand{\thefigure}{\ref{#2}}% Figure counter is \ref
   \renewcommand{\addcontentsline}[3]{}% Avoid placing figure in LoF
   \begin{figure}[#1]}
  {\end{figure}}
  
  \usepackage{titlesec}

%\setcounter{secnumdepth}{4}

\titleformat{\paragraph}
{\normalfont\normalsize\bfseries}{\theparagraph}{1em}{}
\titlespacing*{\paragraph}
{0pt}{3.25ex plus 1ex minus .2ex}{1.5ex plus .2ex}

 \usepackage{geometry}

 
 \newcommand{\E}{\mathbb E}
 \newcommand{\M}{\mathcal M}
\renewcommand{\th}{\hat\theta}
\renewcommand{\t}{\theta}

\begin{document}


\title{Synergistic information\thanks{We would like to thank.}}
\author{Antoine Dubus and Patrick Legros}
\date{\today}

%\author{Antoine Dubus\thanks{~~Université Libre de Bruxelles, ECARES; \href{mailto:antoine1dubus@gmail.com}{antoine1dubus@gmail.com}.}}

\maketitle

\begin{abstract}

\noindent 

\end{abstract}
 
\textbf{Very preliminary, do not circulate.}

\baselineskip0.7cm

\section{Literature}
\begin{itemize}\setlength\itemsep{-1em}
    \item Optimal merger policy when firms have private information \cite{Besanko1993}
    \item Information sharing in oligopolies
    \item Data as assets
    \item Computer science and data synergies
    \item Joint ventures before merger
\end{itemize}

\section{Model}


\begin{itemize}\setlength\itemsep{0em}
	\item Two firms, 1 and 2. 
	\item Firm $1$ is endowed with a data set $X$. Firms make profits using data/ 
	\item Firm 2 want to merge with Firm 1. The merged entity makes profits $\Pi_m=X(1+\t)$ where $\t$ is the synergistic value of information owned by Firm 1 when combined with the data owned by firm $2$. 
	\item Firm 2 and a regulator believe that the type of Firm 1 is uniformly distributed over $[\underline{\t},\overline{\t}]$.
    \item Firm $1$ can share $s$ information from the total data set $X$, which is verifiable by Firm $2$ and the regulator.
	\item The regulator will choose whether to allow the merger or not. The regulator tradeoffs the cost from increased market power (assumed to simplify to be proportional to the industry profit) and the gain from synergies. There is some uncertainty on the weight $w$ the regulator will put on synergies, and this uncertainty is resolved only at the time the regulator evaluates the merger proposal, that is after firms $1,2$ have played their mechanism, and $s$ has been shared.
	\item If the merger fails, the information that has been shared by Firm $1$ can be used by Firm 2 during the competition, and information sharing changes the profit of Firm $1$: $X-\theta l(s)$; and of Firm 2: $\t s$. Hence the total industry profit if there is competition is $X-\t l(s)+\t s$. We assume that
    %
    \[
        l(0)=l'(0)=0
    \]
    %
	\item There are two relevant cases to consider:
\begin{enumerate}[(a)]
    \item (Costly Sharing:) $l(s)\geq 0$: sharing information increases competition and firm $1$ incurs a loss that is increasing in the amount that is shared ($l'(s)>0$).
    \item (Collusive Sharing:) $l(s)< 0$: sharing information facilitates coordination and increases the profits of a firm (à la \cite{vives1984duopoly} for instance) in an increasing way $s$ (hence because $-l(s)$ is the gain, need $l'(s)<0$).
 \end{enumerate}
    \item Firm 2 offers a menu $(s,T(s))$ and firm $1$ chooses -- as a function of her type $\theta$ -- how much to share. Conditional on obverving $s$, both firms decide to approach the regulator to authorize the merger. We assume that there is no renegotiation on $T(s)$ at this stage.
    \item If the merger occurs, Firm $2$ agrees to make a transfer  $T(s)$ to Firm 1 and be residual claimant on the profit.\footnote{%
    This payment scheme is without loss of generality. What matters is that the profit of firm $1$ in case of merger depends on its true type, for otherwise revelation of information is impossible. Also, for incentive compatibility, it is clear that if two $\theta$s share the same $s$, they will claim to be of the type that minimizes $T(\theta,s)$.}
    
\end{itemize}

\paragraph{The timing of the game is the following.} ($U[a,b]$ denotes the uniform distribution on $[a,b]$; we could use more general distributions...)
\begin{enumerate}\setlength\itemsep{0em}
    \item Firm $2$ offers a rmenu $(s,T(s)$, $s$ being the sharing of data that firm $1$ should do and $T(s)$ is the price firm $2$ agrees to pay to firm $1$ if the merger is agreed and the regulator authorizes it.
    \item Firm 1 privately learns its type $\t\sim U[\underline{\t},\overline{\t}]$ and makes an announcement in the mechanism and plays $s$.
    \item If both firms anticipate to be better off via a merger, they approach the regulator  and ask for the merger to be approved. At this point, the regulator sees $s,T(s)$.
    \item The regulator observes a draw $w$ from a distribution $F$ on $\mathbb R_+$ with a continuous density. The regulator decides to allow or to prevent the merger. The market structure, profits and welfare, are realized.
\end{enumerate}
%
\subsection{The Regulator's Problem}
   The regulator maximizes a social welfare function that weights the social cost of high industry profits and the social benefit of synergies. Contrary to the usual view that synergies are created only during the merger, synergies endogenously happen without a merger if firm $1$ shares some of its data with firm $2$. Hence, when evaluating a merger proposal, the regulator will compare the \emph{relative synergy gain} to the \emph{relative industry profit gain.}

   At the time the regulator has to evaluate a merger, firm $1$ has already shared $s$ with firm $2$, following a strategy $\sigma(\t)$ in the sharing game.  Let $\sigma^{-1}=\{\t|\sigma(\t)=s\}$ be the set of types of firm $1$ consistent with a sharing of $s$ given the equilibrium strategy $\sigma$. Hence, conditional on observing $s$, firm $2$ and the regulator believe that the expected synergy is the conditional expectation $$\t_s:=\E[\t|\t\in\sigma^{-1}(s)].$$
     
    If the merger is allowed and firm $2$ takes control of firm $1$, the loss from industry profit is $-[X(1+\t)]$ while the synergy gain is $w[X(1+\t)]$, hence welfare is
    %
    $$-X+(w-1)X(1+\t_s).$$
    %
    If the merger is prevented or the firms decide not to go ahead with it, firm $2$ benefits from the information given by firm $1$ (by a factor of $\t s$, where $\t$ is the true state of the world), and firm $1$ has a loss of $l(s)$, implying that the total industry profit increases by $\t s -l(s)$. Therefore, the regulator believes that the realized industry profit will be $X-\t_s(l(s)-s)$   if he does not authorize the merger, while the synergy benefit would be $w[X+\t_s s]$.\footnote{%
    As it should be, the regulator ignores the market loss $l(s)$ to firm $1$.} Hence the expected welfare under competition is
      $$-X+\theta_s l(s)-\t_s s + w (X+\t_s s)$$
      Our specification implies therefore that the probability that the merger is authorized depdns only on the share $s$ but not on the beliefs of the regulator about $\t$. Indeed,welfare is greater with than without the merger when 
   %    
    \begin{equation}
           w\geq w^*(s):=1 + \frac{l(s)}{X-s}
    \end{equation}
    %


At the ex-ante stage, when firms play their mechanism to share information, the probability that a merger will be approved when firm $1$ shares $s$ with firm $2$ is then equal to  
\[
a(s):=1-F(w^*(s)).
\]

\section{Playing the Regulatory Game with Costly Sharing}
At the time firms $1,2$ play the mechanism, the probabilities $a(s)$  are taken as given because if called upon to act, the regulator has sunk beliefs (he believes that types in $\sigma^{-1}(s)$ of firm $1$ have shared $s$.) Firm $2$ will ask for a merger only if when firm $1$ selects an element of the offered menu. By the revelation principle, there is no loss in considering menus
\[
\M:=\{(s(\t),T(\t)),\theta\in[\underline \t,\overline \t]\}.
\]
%
\paragraph{The Incentives to Share.}
Let $s(\t)$ denotes the choice of sharing by firm $1$ of type $\t$. There will be a merger with probability $a(s(\t))$, hence firm 1's expected profit is
\[
U(\t):= a(s(\t))(X(1+\t)-T(\t)+(1-a(s(\t)))(X_1-l(s(\t))),
\]
while by deviating to $s\neq s(\t)$ there is competition for sure and firm $1$'s payoff is $X-l(s)$. In the case of sharing being competitive, the best deviation is to use $s=0$, and therefore the individual rationality of firm $1$ is 
%
\begin{equation}\label{cond:IR}
    a(s(\t)(X(1+\t)-T(\t))+(1-a(s)))(X-\t l(s(\t)))\geq X.   \tag{IR1}
\end{equation}
%
If type $\t$ selects another element of $\makeatletter$, he will choose to follow the recommendation $s(\th)$ when the IR condition for type $\th$ is satisfied since by deviating to $\th$ and doing $s(\th)$, type $\t$ gets
%
  \begin{align*}
    U(\th|\t)&:=a(s(\th))(X(1+\t)-T(\th))+(1-a(s(\th)))(X-l(s(\th))).
  \end{align*}
%
It remains therefore to verify that type $\t$ does not want to deviate to $\th$ and follow the recommended shares, that is that $U(\t)\geq U(\th|\t)$. Now,
\begin{align*}
U(\th|\t)=U(\th)+a(s(\th))X(\t-\th)
\end{align*}
%
and therefore the two incentive constraints $U(\th|\t)\leq U(\t)$ and $U(\t|\th)\geq U(\th)$ yield
\[
a(s(\th))X(\t-\th)\leq U(\t)-U(\th)\leq a(s(\t))X(\t-\th)
\]
Therefore, $a(s(\t))$ and $U(\t)$ are non-decreasing functions. Because sharing of information is costly for firm $1$, $l(s)>0$ and $l'(s)>0$. Then, the ratio $\frac{l(s)}{X-s}$ is an increasing function of $s$ and therefore $a(s)$ is a decreasing function of $s$. Therefore it is possible that $a(s(\t))$ is non-decreasing in $\t$ only if $s(\t)$ is non-increasing in $\t$. Hence, higher types should \emph{share less}.\footnote{
If instead sharing of information is collusive that is $l(s)<0$ and $l'(s)<0$, then $a(s)$ is an increasing function of $s$ and $a(s(\t))$ is non-decreasing in $\t$ only if $s(\t)$ is non-decreasing in $\t$. Hence, higher types should \emph{share more}. 
}

%
\todo[inline]{Quid if firms know that information is competitive or collusive but the regulator observes only $s$ but does not know if this is helping or hindering competition?}

\paragraph{Unique price.} Let us first consider the benchmark when firms do not share information. Then $\t_0$ is the expected value of $\t$ and firm $2$ offers a take-it-or-leave-it offer to firm $1$ to merge at a price of $T_0$ with a prior sharing $s_0$. The individual rationality condition \eqref{cond:IR} of firm $1$ is $T_0\leq \t (X+l(s_0))$. Given this, firm $2$ has an expected surplus of $a(s_0)T_0+(1-a(s_0))\theta_{s_0}s_0$, where $\theta_{s_0}=\E[\theta|\theta\geq T_0/(X+l(s_0))]$. 
%






\bibliographystyle{plain}
\bibliography{Bibliography}





\end{document}

