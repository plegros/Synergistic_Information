%  Synergies and mergers-Thursday, 19 November 2020
%
%  Created by Patrick on 2020-11-19.
%  Copyright (c) 2020 __Patrick Legros__. All rights reserved.
%
%!TEX encoding = UTF-8 Unicode  

\documentclass[a4paper]{article}
\usepackage[utf8]{inputenc}
\usepackage{amsthm,amsmath,amssymb,amsfonts}
\usepackage{url}
\usepackage{hyperref}
\usepackage[round]{natbib}
%\bibliographystyle{plainnat}
\usepackage[mathscr]{euscript}
\let\euscr\mathscr \let\mathscr\relax
\usepackage[scr]{rsfso}
\usepackage{graphicx}
\usepackage{float}
\usepackage{enumerate}
\usepackage{blindtext}
\usepackage{setspace}
\usepackage{eurosym}
\usepackage{hyperref}
\usepackage{pdflscape}
\usepackage{array, multirow}
\usepackage[dvipsnames]{xcolor}

\usepackage{todonotes}

\usepackage{tikz}
\hypersetup{
colorlinks,
citecolor=NavyBlue,
linkcolor=NavyBlue,
urlcolor=NavyBlue}
\usetikzlibrary{matrix}
\newtheorem{prop}{Proposition}
\newtheorem{lemma}{Lemma}
\newtheorem{corollary}{Corollary}
\newtheorem{ass}{Assumption}
\newtheorem{result}{Result}
\newtheorem{condition}{Condition}
\newtheorem{Theorem}{Theorem}
\newtheorem{Definition}{Definition}
\newtheorem{remark}{Remark}

\newenvironment{reusefigure}[2][htbp]
  {\addtocounter{figure}{-1}%
   \renewcommand{\theHfigure}{dupe-fig}% If you're using hyperref
   \renewcommand{\thefigure}{\ref{#2}}% Figure counter is \ref
   \renewcommand{\addcontentsline}[3]{}% Avoid placing figure in LoF
   \begin{figure}[#1]}
  {\end{figure}}
  
  \usepackage{titlesec}

%\setcounter{secnumdepth}{4}

\titleformat{\paragraph}
{\normalfont\normalsize\bfseries}{\theparagraph}{1em}{}
\titlespacing*{\paragraph}
{0pt}{3.25ex plus 1ex minus .2ex}{1.5ex plus .2ex}

 \usepackage{geometry}

 
 \newcommand{\E}{\mathbb E}
 \newcommand{\N}{\mathcal N}
\renewcommand{\th}{\hat\theta}
\renewcommand{\t}{\theta}
\renewcommand{\a}{\alpha}
\renewcommand{\l}{\lambda}


\begin{document}


\title{Synergistic Information$^{3.0}$}
\author{Antoine Dubus and Patrick Legros\thanks{We would like to thank.}}
\date{\today}

%\author{Antoine Dubus\thanks{~~Université Libre de Bruxelles, ECARES; \href{mailto:antoine1dubus@gmail.com}{antoine1dubus@gmail.com}.}}

\maketitle

\begin{abstract}

\noindent 

\end{abstract}
 
\textbf{Very preliminary, do not circulate.}

\baselineskip0.7cm

\section{Literature}
\begin{itemize}\setlength\itemsep{-1em}
    \item Competition policy for the digital era
    
    \cite{tirole2020competition}; \cite{scott2019committee}; \cite{cremer2019competition}; \cite{cabral2020merger}
    
    None of them has seen data as a motivation to acquire merger. We fill a gap in this literature.
    
    \item Optimal merger policy when firms have private information \cite{Besanko1993}
    \item Information sharing in oligopolies
    
    \cite{vives1984duopoly, gal1986information} sharing information can have pro or anti competitive effects depending on the nature of competition (cournot vs bertrand)
    
    They justify that we look at $l$ positive and negative
    \item Disclosure to consumers 
    \item Data as assets
    
    \cite{stucke2016introduction} discuss how mergers are now motivated by the acquisition of a firm data set
    \item Computer science and data synergies:
    
    \cite{bertschinger2014quantifying, Griffith2014, olbrich2015information} discuss how information synergies can arise when merging two data sets, 
    
    justify why we focus on information synergies
    \item Joint ventures before merger
    \item incomplete contracts and cooperative investments \citep{Che1999}
\end{itemize}



\section{Concrete example}

\begin{itemize}
    \item pre merger information sharing
    
    \href{https://sites-herbertsmithfreehills.vuturevx.com/46/12874/compose-email/the-altice-case--a-costly-warning-not-to-engage-in-gun-jumping-before-receiving-merger-control-clearance.asp}{Altice/OTL} The FCA found that Altice and SFR engaged in an extensive exchange of commercially sensitive information (including individualised trade data and future forecasts)
    
    \href{https://www.twobirds.com/en/news/articles/2020/global/double-caution-gun-jumping-risks-in-m-and-a-transactions}{Gun-jumping examples}
    \item Acquisition failures:
    
    \href{https://www.investopedia.com/articles/insights/061816/4-cases-when-ma-strategy-failed-acquirer-ebay-bac.asp}{ebay/skype}: lack of complementarity
    
    \href{https://salessynergy.net/the-biggest-acquisition-disasters-that-put-companies-into-quite-a-bit-of-trouble/}{google/motorola}: android bugs on device
    
    \href{https://www.investopedia.com/articles/financial-theory/08/merger-acquisition-disasters.asp#:~:text=The\%20consolidation\%20of\%20AOL\%20Time,combination\%20up\%20until\%20that\%20time}{sprint/nextel} lack of common culture
    
    \item Regulator forbids merger:
    
    \href{https://www.livemint.com/companies/news/aurobindo-pharma-calls-off-1-billion-deal-with-sandoz-after-failing-to-get-ftc-nod-11585801128011.html}{Aurobindo/sandoz}
    
    \item Profit sharing mechanisms between two firms: \href{https://www.nytimes.com/2020/11/21/us/politics/coronavirus-vaccine.html?referringSource=articleShare}{Politics, Science and the Remarkable Race for a Coronavirus Vaccine}.
\end{itemize}





\section{Introduction}
Idea: firm $1$ with dataset may have a merger opportunity in the future with another firm $2$ using also data. If there are synergies, the merger is beneficial, but without prior information, firm $2$ may be reluctant to merge. Revelation of information about synergies can be done in one of two ways:
\begin{itemize}
  \item By prior sharing of some of the data from firm $1$. If the amount of data that is shared is sufficiently large, this will enable firm $2$ to learn about the synergy level; otherwise there is no learning. For the example below, the assumption is that sharing below $s^*$ does not bring information, but sharing about $s^*$ brings information. A more `continuous' model will have a change in the precision of information continuous as a function of $s$.
  \item Or by waiting and bargaining under incomplete information on the part of firm $2$ about the level of the synergy. 
\end{itemize}
Hence, waiting to merge generates ex-post inefficiencies while not waiting insures ex-post efficiency (since there is symmetric information at the time of the merger). From firm $1$'s point of view, sharing of data brings a competitive disadvantage if the merger fails and firms compete (the case of sharing for collusion to be analyzed next). Hence, to induce firm $1$ to share, firm $2$ has to offer a high price for the shared data, and give rents to firms when the synergy is low. By contrast, waiting and bargaining under asymmetric information gives rents to firms $1$ when the synergy is high.


\section{Model}
\begin{itemize}
    \item Firm 2 can acquire firm 1 without knowing its type:
    \begin{itemize}
        \item Firm 2 proposes a menu composed by couple $(\a(\t),p(\t))$. 
        \item Firm 1 selects $(\a(\t),p(\t))$.
        \item Firm 2 transfers $p(\t)$, and believes Firm 1 has type $\t$.
        \item Firm 2 then either decides to go on with the merger, invests $c$, give $\a(\t)$ to firm 1 and profits are $(1-\a(\t))[(1+\t)X-c]-p(\t)$ for firm 2 and $\a(\t)[(1+\t)X-c]+p(\t)$ for firm 1.
        \item Or firm 2 does not pursue and merger does not occur, Firm 1 gets $p(\t)+X$ and firms 2 makes $-p(\t)$.
    \end{itemize}   
  \item Or firm 2 can buy $s$ information from firm $1$ and learn its type:
  \begin{itemize}
      \item Either firm 2 does not exploit the data, firm 1 makes $X$ and firm 2 makes $0$
      \item Or firm 2 exploits the data at cost $c$, firms compete, the profits are $X-L(\t)$ for $1$ and $\t s-c$ for $2$.
      \item Or firms merge, exploit the data at cost $c$ and the industry profits are $X(1+\t)-c$
  \end{itemize}
  \item Firm $2$ has full negotiation power.
  \item There are two relevant levels of sharing: $s=0$ and $s^*$. If $0$, firm $2$ does not know $\t$ at the time of merger; if sharing is $s^*$, firm $2$ knows $\t$.
  \item $\t$ has ex-ante distribution $F(\t)$, with continuous density and support $[0,\overline \t]$ ($\overline \t=\infty$ is allowed), MLRP satisfied.
  \item We assume that cost $c$ for firm 2 to exploit the data is never higher than the profits of firm 1: $X\geq c$
\end{itemize}


\paragraph{Steps to follow:}
\begin{enumerate}[(1)]\setlength\itemsep{0em}
  \item Let $T$ the price that firm $2$ offers to pay for data $s^*$.
  \item If firm $1$ accepts, $s^*$ is shared and firm $2$ learns $\t$. 
  \begin{itemize}
      \item For $\t \geq \frac{c}{s}$ it is profitable for firm 2 to merge or to exploit the data, merger is preferred to exploiting
      \item For $\t \leq \frac{c}{s}$ firm 2 go on with the merger but does not find it profitable to exploit the data
      \item for $\t \leq \frac{c}{X}$ firm 2 refuses merge or to exploit
  \end{itemize}  
  When merger opportunity arises, firm $2$ knows the type of firm 1 and extracts all the surplus from merger. Let $U_1(\t|s=s^*)$ be the expected utility of type $\t$ of doing data sharing.
  \item Let $\N$ be the subset of types that decline $T$.
  \item\label{stage-merger} Given $\N$, there is an optimal revelation mechanism that maximizes firm $2$'s expected payoff subject to the IR and IC conditions for firm $1$. Let $U(\t|s=0)$ be the expected payoff of firm $1$ of type $\t$ when playing this mechanism (corresponding to the `sunk' belief of firm $2$ of facing firms $1$ with types in $\N$. Note that we can compute this value for any $\t$, which is obviously necessary in order to verify the incentives of different types to do data sharing or not.)
  \item Go back and verify that for each $\t\in \N$, $U_1(\t|s=0)\geq U_1(\t|s=s^*)$ and for each $\t\notin \N$, $U_1(\t|s=0)\leq U_1(\t|s=s^*)$
\end{enumerate}
%
Note that we assume that firm $2$ does not commit to the mechanism used at stage \eqref{stage-merger}. Probably not necessary, but facilitates derivations; also quite relevant in the \cite{anton2002sale} type of environment.

As an illustration of the mechanics of the model, suppose firm $2$ has full bargaining power. Firms that accept to share $s^*$ for a price of $T$ anticipate that firm $2$ will make a TIOLI offer if a merger possibly arises and will extract all the surplus. Hence, the payoff of firms that accept the offer to share is   
%

\begin{itemize}
    \item If $\t\in[0,\frac{c}{X}]$, neither merger nor competition are profitable for firm 2 and
    \[U_1(\t|s=s^*)=X+T\]
    \item If $\t\in[\frac{c}{X},\frac{c}{s}]$ firm 2 will always want to merge but does not want to compete with firm 1, thus the profit of firm 1 is
    
    \[U_1(\t|s=s^*)=X+T\]
    \item If $\t\in[\frac{c}{s},\overline \t]$ firm 2 will always want to merge and want to compete with firm 1 if merger fails, thus the profit of firm 1 is
    
    \[U_1(\t|s=s^*)=X+T-L(\t)\]
\end{itemize}

%

which is a decreasing function of $\t$. By contrast, firms that do not share data anticipate that firm $2$ will make an offer that gives them an informational rent, that is $U_1(\t|s=0)$ is increasing in $\t$. As we will see, this is a standard screening problem and, because of a lack of commitment of firm $2$, firm $1$ anticipates that the participation constraint of the lowest type $\t_0$ who does not share data will be binding and that all types greater than $\t_0$ get a rent. It follows that $\N$ is an interval $[\theta_0,\overline \t]$. Furthermore, because $\t_0$ must be indifferent between sharing and not sharing data, we need $T+X-L(\t_0)=X$, or
\begin{equation}\label{eq:T}
  T=L(\t_0).
\end{equation}
%

\paragraph{Consider $\t\in[\frac{c}{X},\frac{c}{s}]$}


Following sharing by type $\t$, if a merger opportunity arises, firm $1$ will accept the merger if she receives a payoff of $X$, which corresponds to its outside option (transfer $T$ is sunk). Therefore, the expected payoff to firm $2$ of inducing sharing by type $\t$ is equal to $X(1+\t)-X-L(\t_0)-c=\t X-L(\t_0)-c$. Firm $1$ has expected payoff $X+L(\t_0)$.


\paragraph{Consider $\t\in[\frac{c}{s},\overline \t]$}


Following sharing by type $\t$, if a merger opportunity arises, firm $1$ will accept the merger if she receives a payoff of $X-L(\t)$, because if it declines the offer, firm 2 now has interest to compete. Therefore, the expected payoff to firm $2$ of inducing sharing by type $\t$ is equal to $X(1+\t)-(X-L(\t))-L(\t_0)-c=\t X +L(\t)-L(\t_0)-c$. Firm $1$ has expected payoff $X-L(\t)+L(\t_0)$.


\paragraph{Mechanism if firm $2$ believes that $\t\in\N$ at the merging phase.} At the merging stage, firm $2$ believes that types have a distribution with support on $\N=[\t_0,\overline \t]$. Note that all types in $\N$ do not share data and have the same outside option of $X$. The value to the merger is $X(1+\t)-c$. 

A mechanism is a menu $\{(p(\t),\a(\t));\t\in \N\}$, where $p(\t)$ is the price paid by firm $1$ to firm $2$ and $\a(\t)$ is the share of the merged entity that firm $2$ gives to firm 1.\footnote{Clearly firm 2 would prefer to give $p=X$ to all firms and get all the surplus. However, in this case firm 1 of type $\t\in[0,\frac{c}{X}]$ has interest not to sell information, and to bargain under asymmetric information. Indeed in this case firm 2 will pay $p$ but wont find it profitable to merge, and thus firm 1 will make $X+p>X+L(\t_0)$. Thus firm 2 cannot set $p>L(\t_0)$ and has to set $\a >0$ so that firm 1 accepts the offer.} For firm 1 to accept the menu it is necessary that firm 2 commits to giving share $\a$ to firm 1 if the merger occur, else firm 2 could give a small transfer, obtain all the data from firm 1 and exploit it without merging. It should be clear that if $\t_0$ does not get a rent in the mechanism, types $\t<\t_0$ get a negative rent if they do not share data; by contrast they get a positive rent equal to $L(\t_0)-L(\t)$ if they share data. The participation constraint of firm $1$ is
\begin{equation}
  U(\t):=p(\t)+\a(\t)[X(1+\t)-c]\geq X 
\end{equation}

while the truth-telling constraint is
\begin{equation*}
  \t \in \arg\max_{\th} U(\th|\t):=p(\th)+\a(\hat \t)[X(1+\t)-c]
\end{equation*}
%
Usual manipulations yield 
%
\[
  X \a(\th)(\t-\th)\leq U(\t)-U(\th)\leq X\a(\t)(\t-\th)
\]
%
hence that $U(\t)$ and $\a(\t)$ are almost everywhere non-decreasing function. Moreover, by the envelop theorem, $\dot U(\theta)=\a(\t)X$. 

Hence, firm $2$ offers a mechanism $(p,\a)$ to solve $\max_{\{p(\cdot),\a(\cdot)\}}\int_{\t_0}^{\overline \t}\left[(1-\a(\t))[(1+\t)X-c]-p(\t)\right]f(\t)d\t$ subject to the IR and IC constraints. By using $-p(\t)=\a(\t)[(1+\t)X-c]-U(\t)$, the problem can be rewritten as
\begin{align*}
  \max_{(p(\cdot);\a(\dot))} \int_{\t_0}^{\overline \t} \left(-U(\t)+X(1+\t) \right)&\frac{f(\t)}{1-F(\t_0)}d\t\\ 
  U(\t) \geq& X  \hspace{2cm} \text{(IR)} \\ 
  \dot{U}(\t)=&\a(\t)X \hspace{2cm} \text{(IC)} \\ 
\end{align*}
%
Standard derivations yield to the equivalent problem 
%
\begin{align*}
\max_{\{\a(\t)\}} &\int_{\t\geq \frac{c}{X}\in \N}\left( (1+\t)X-\a(\t)X\frac{1-F(\t)}{f(\t)}\right)f(\t)d\t.
\end{align*}
%
Which is maximized when $\a(\t)$ is minimized.
\todo[inline]{Patrick là je ne suis pas sur: This is achieved for $\a=\a(\t_0)=\frac{X-L(\t_0)}{(1+\t_0)X-c}$. Si je fais ca on n'est plus dans du screening mais on a quand meme bien exclusion des bas types}
%
The expected payoff to firm $2$ is (using $p=L(\t_0)$ and $\a=\frac{X-L(\t_0)}{(1+\t_0)X-c}$)
%

\paragraph{When $\frac{c}{X}\geq\t_0$}

Firm 2 faces an adverse selection problem: all firm 1 of type $\t\in[0,\frac{c}{X}]$ have interest to accept the menu, take highest transfer $\overline p$ as they know that the merger will not be carried on by firm 2. Thus the profits of firm 2 is:


\[
V(\t_0):=\int_{0}^{\frac{c}{X}}-\overline p f(\t)d\t+\int_{\frac{c}{X}}^{\overline \t}\left[(1+\t)X-c-\a(\t)X\frac{1-F(\t)}{f(\t)}\right]f(\t)d\t
\]

\footnote{\[
V(\t_0):=\int_{0}^{\frac{c}{X}}-L(\t_0)f(\t)d\t+\int_{\frac{c}{X}}^{\overline \t}\left[(1+\t)X-c-\frac{X-L(\t_0)}{1+\t_0-\frac{c}{X}}\frac{1-F(\t)}{f(\t)}\right]f(\t)d\t
\]}




Nevertheless, larger $p$ decrease the information rent given to firm 1 on $[\frac{c}{X},\overline \t]$: 

\[
V'(\t_0):=-L'(\t_0)F(\frac{c}{X})+\frac{L'(\t_0)(1+\t_0-\frac{c}{X})+(X-L(\t_0))}{(1+\t_0-\frac{c}{X})^2}\int_{\frac{c}{X}}^{\overline \t}1-F(\t)d\t
\]




\[
V'(0)\geq0 \iff -L'(0)F(\frac{c}{X})+\frac{L'(0)(1-\frac{c}{X})+X}{(1-\frac{c}{X})^2}\int_{\frac{c}{X}}^{\overline \t}1-F(\t)d\t
\]

\[
\int_{\frac{c}{X}}^{\overline \t}1-F(\t)d\t=E_{\frac{c}{X}}[\t]-\frac{c}{X}(1-F(\frac{c}{X}))
\]

Uniform distribution over $[0,\overline \t]$:

\[
V'(0)\geq0 \iff L'(0)(\frac{1}{2\overline \t}\frac{(\overline \t-\frac{c}{X})^2}{1-\frac{c}{X}}-\frac{c}{X\overline \t})+X\frac{1}{2\overline \t}\frac{(\overline \t-\frac{c}{X})^2}{(1-\frac{c}{X})^2}
\]

Exponential distribution with PDF $f(\t)=e^{-\l \t}$:

\[
V'(0)\geq0 \iff L'(0)(\frac{1}{\l}\frac{e^{-\l c/X}}{1-\frac{c}{X}}-1+e^{-\l c/X})+X\frac{e^{-\l c/X}}{\l(1-\frac{c}{X})^2}
\]



\paragraph{When $\frac{c}{s}\geq\t_0\geq\frac{c}{X}$}


Consumers on $[0,\frac{c}{X}]$ will behave as before and firm 2 will incur a loss of profits of $-L(\t_0)$. Consumers on $[\t_0,\overline \t]$ will accept the merger without information sharing.

Consumers on $[\frac{c}{X},\t_0]$ will not accept the merger under asymmetric information. However firm 2 can purchase information from these firms. As their outside option is not to merge and get $X$, firm 2 can set a price to zero for information, learn the type of firm 1 and merge by proposing a transfer equal to $X$. In this case the profit of firm 2 is $\t X-c$. Thus the total profit of firm 2 in this case is:



\[
V(\t_0):=
    \int_0^{\frac{c}{X}} -L(\t_0)f(\t)d\t +\int_{\frac{c}{X}}^{\t_0}(\t X-c)f(\t)d\t+\int_{\t_0}^{\overline \t}\left[(1+\t)X-c-\frac{X-L(\t_0)}{1+\t_0-\frac{c}{X}}\frac{1-F(\t)}{f(\t)}\right]f(\t)d\t
\]


\[
V'(\t_0):=-L'(\t_0)F(\frac{c}{X})-X f(\t_0)+\frac{X-L(\t_0)}{1+\t_0-\frac{c}{X}}(1-F(\t_0))+\frac{L'(\t_0)(1+\t_0-\frac{c}{X})+(X-L(\t_0))}{(1+\t_0-\frac{c}{X})^2}\int_{\t_0}^{\overline \t}1-F(\t)d\t
\]

Firm 2 chooses $\t_0\geq \frac{c}{X}$ if $V'(\frac{c}{X})\geq0$:

\[
V'(\frac{c}{X})\geq0 \iff-L'(\frac{c}{X})F(\frac{c}{X})-X f(\frac{c}{X})+(X-L(\frac{c}{X}))(1-F(\frac{c}{X}))+(L'(\frac{c}{X})+X-L(\frac{c}{X}))\int_{\frac{c}{X}}^{\overline \t}1-F(\t)d\t\geq0
\]

Uniform distribution over $[0,\overline \t]$:

\[
V'(\frac{c}{X})\geq0 \iff L'(\frac{c}{X})(\frac{\overline{\t}}{2}-\frac{c}{X}+\frac{c^2}{2X^2\overline \t}-\frac{c}{X\overline \t})+(X-L(\frac{c}{X}))(\frac{1}{\overline \t}(\overline{\t}-\frac{c}{X})+\frac{1}{2\overline{\t}}(\overline{\t}-\frac{c}{X})^2)-\frac{X}{\overline{\t}}\geq0
\]

Exponential distribution with PDF $f(\t)=e^{-\l \t}$:


\[
V'(\frac{c}{X})= L'(\frac{c}{X})((1+\l)e^{-\l c/X}-1)+X(1-\l+\frac{1}{\l}) e^{-\l c/X}-(1+\frac{1}{\l})L(\frac{c}{X})e^{-\l c/X}
\]


\paragraph{When $\t_0\geq\frac{c}{s}$}


There are now firm 1 with types $\t\in[\frac{c}{s},\t_0]$, that will not sell $s$ at a zero price. Indeed, if firm 2 purchase information from these firms, it will want to exploit the data purchased in case there is no merger, and the outside option of firm 1 is $X-L(\t)$. Thus the minimum price that firm 1 in this interval accepts to sell data is $L(\t)$, which is set at $L(\t_0)$ so that all types up to $\t_0$ sell information. Profits of firm 1 are $X+L(\t_0)-L(\t)$, and for firm 2 $\t X-L(\t_0)+L(\t)$

Thus consumers on $[0,\frac{c}{X}]$ will behave as before - and are now indifferent between selling information at price $L(\t_0)$ or engaging negotiation for a merger that will not occur in the end - and firm 2 will incur a loss of profits of $-L(\t_0)$. Consumers on $[\t_0,\overline \t]$ will accept the merger without information sharing.

Consumers on $[\frac{c}{X},\t_0]$ will not accept the merger under asymmetric information. However firm 2 can purchase information from these firms, and they are now pooled with types in $[\frac{c}{s},\t_0]$. In this case the profit of firm 2 is $\t X-c-L(\t_0)$. Thus the total profit of firm 2 in this case is:



\begin{equation}
\begin{aligned}
  V(\t_0):=&\int_0^{\frac{c}{X}} -L(\t_0)f(\t)d\t +\int_{\frac{c}{X}}^{\frac{c}{s}}(\t X-c-L(\t_0))f(\t)d\t+\int_{\frac{c}{s}}^{\t_0}(\t X-L(\t_0)+L(\t)-c)f(\t) d\t\\
  &+\int_{\t_0}^{\overline \t}\left[(1+\t)X-c-\frac{X-L(\t_0)}{1+\t_0-\frac{c}{X}}\frac{1-F(\t)}{f(\t)}\right]f(\t)d\t
\end{aligned}
\end{equation}


\begin{equation}
\begin{aligned}
  V'(\t_0):=&-L'(\t_0)F(\t_0)-X f(\t_0)+\frac{X-L(\t_0)}{1+\t_0-\frac{c}{X}}(1-F(\t_0))\\
  &+\frac{L'(\t_0)(1+\t_0-\frac{c}{X})+(X-L(\t_0))}{(1+\t_0-\frac{c}{X})^2}(E_{\t_0}(\t)-\t_0(1-F(\t_0)))
\end{aligned}
\end{equation}


\paragraph{Does information sharing occur in equilibrium?}

A sufficient condition for information sharing to occur, is that profits always increase on $[0,\t_0^*]$ with $\t_0^*>\frac{c}{X}$

\medskip

On $[0,\frac{c}{X}]$: 

\[
V'(\t_0):=-L'(\t_0)F(\frac{c}{X})+\frac{L'(\t_0)(1+\t_0-\frac{c}{X})+(X-L(\t_0))}{(1+\t_0-\frac{c}{X})^2}\int_{\frac{c}{X}}^{\overline \t}1-F(\t)d\t
\]

Clearly a sufficient condition for this expression to be positive is $\int_{\frac{c}{X}}^{\overline \t}1-F(\t)d\t\geq F(\frac{c}{X})$

which is satisfied for $E[\t|\t\geq \frac{c}{X}]\geq 1$

\medskip

On $\frac{c}{s}\geq\t_0\geq\frac{c}{X}$

\[
V'(\t_0):=-L'(\t_0)F(\frac{c}{X})-X f(\t_0)+\frac{X-L(\t_0)}{1+\t_0-\frac{c}{X}}(1-F(\t_0))+\frac{L'(\t_0)(1+\t_0-\frac{c}{X})+(X-L(\t_0))}{(1+\t_0-\frac{c}{X})^2}\int_{\t_0}^{\overline \t}1-F(\t)d\t
\]

A sufficient condition for this expression to be positive is 

$E[\t|\t\geq \frac{c}{s}]\geq F(\frac{c}{X})(1+\frac{c}{s}-\frac{c}{X})$ and $\frac{X-L(\frac{c}{S})}{1+\frac{c}{S}-\frac{c}{X}}\frac{1-F(\frac{c}{S})}{f(\frac{c}{S})}\geq X$

\paragraph{Does costly information sharing occur in equilibrium?}

A sufficient condition for costly information sharing to occur is that both conditions above are verified and that $V'(\frac{c}{s})>0$:

\medskip

On $\t_0\geq\frac{c}{s}$
\begin{equation}
\begin{aligned}
  V'(\t_0):=&-L'(\t_0)F(\t_0)-X f(\t_0)+\frac{X-L(\t_0)}{1+\t_0-\frac{c}{X}}(1-F(\t_0))\\
  &+\frac{L'(\t_0)(1+\t_0-\frac{c}{X})+(X-L(\t_0))}{(1+\t_0-\frac{c}{X})^2}(E_{\t_0}(\t)-\t_0(1-F(\t_0)))
\end{aligned}
\end{equation}



Information sharing for types $\t_0\geq \frac{c}{s}$ occur if $V'(\frac{c}{s})\geq 0$

\begin{prop}~~
  \begin{enumerate}[(i)]\setlength\itemsep{0em}
    \item when $V'(\frac{c}{s})>0$ information sharing occurs in equilibrium.
    \item There is a set of firms, those with the highest type on $[\t_0,\overline \t]$, that never share information, and for which merger occurs.
    \item Firm 1 with type on $[0,\min\{\t_0,\frac{c}{X}\}]$ are indifferent between selling data and accepting the merger, firm 2 always make transfers $L(\t_0)$ without exploiting the data or merging.
  \end{enumerate}
\end{prop}

\begin{proof}

\end{proof}

\begin{remark}
  The size of the competitive loss $L(\t)$ is key. For small $L$ information sharing will occur for a larger set of parameters. On the opposite, for large $L$... Interpret this in light of potential data sharing between Biotechs and Big pharmas.

  \underline{We need to provide interpretations}, real life examples that could fit with the theoretical results. 
  \end{remark}
  
  
\section{Extension: bilateral information sharing}

\textit{If information is synergistic, it may be possible for firm 2 to share its data with firm 1 (this may also solve the question of asymmetric information post-experimentation in the second model), but may also be a way to have sharing in equilibrium in the model without having a price for sharing data (hence more like a cross-sharing agreement). May be worth looking at in an extension (that supposes that firm 1 has a relative low costs c1 of implementing the synergy).}


\subsection{No competitive loss for firm 2 due to information sharing}


Consider now firm 2 that also has information with synergistic value $\t_2$ (we still denote by $\t$ the synergistic value of information owned by firm 1), whose synergistic value depend on information owned by firm 1: $\t_2 \t$. In this extension firm 1 knows the type $\t_2$ of firm 2 and information sharing is not used by firm 2 to correct an information asymmetry but to separate firm 1 of different types.  

Firm 2 either purchases information from firm 1, or exchanges it against a share $s^*$ of its own information. 

Why would some firms want information instead of money?

The utility of firm 1 of type $\t$ when exchanging information with firm 2 is

\[
X-L(\t)+\t_2 \t
\]

While its profits when selling information is:

\[
X-L(\t)+T
\]

Where $T$ is the price of information. Sharing information is more profitable than selling it for $$\t\geq \frac{T}{\t_2}$$


In case firm 2 sets $T=0$ and there is no information purchasing, only firm 1 with types such that $\t \t_2\geq L(\t)$ accept to share (the fact that all types above, and no type below accept information exchange comes naturally from the concavity of $L$). We write $\underline \t$ that verifies $\t^* \t_2= L(\t^*)$, or equivalently $\t_2= \frac{L(\t^*)}{\t^*}$. The right hand term is always strictly decreasing which guarantees the uniqueness of $t^*$.


Sharing information allows firm 2 to identify low types, in particular $\t\in[0,\frac{c}{X}]$ for which merger is never profitable. However a rent is given to high types, which induces a loss for firm 2.

\subsection{Firm 2 looses $H(\t_2)$ when sharing}

Now by sharing information with firm 1 of type $\t$, firm 2 incurs a cost that we assume to be equal to that of firm 1:

$-L(\t_2)$



\section{EXTENSIONS TO BE DONE}  
(order not necessarily sequential)
    \begin{enumerate}[TBD 1.]\setlength\itemsep{0em}
    \item Collusive data sharing
      \item Look at the case where there is a competitive \emph{gain} of data sharing, that is $l<0$. Firm $1$ is less reluctant to share with firm $2$, so...?
      \item Look at case where sharing $s$ allows firm $2$ to learn the true value of $\t$ with probability $\alpha(s)$, an increasing function of $s$. (The case above coincides with $\alpha(s)=0$ is $s<s^*$ and $\alpha(s)=1$ if $s\geq s^*$.)
      \item The general case where firm $2$ has bargaining power $\beta<1$. Hence it is as if the two firms agree on a price $T$ and a mechanism that satisfies IR and IC for firm $1$ in order to maximize the weighted sum $(1_\beta)U_1+\beta U_2$, that is $U_1+U_2-\beta U_1$: as $\beta$ decreases. Is it less likely that the firms will \emph{not} merge?
      \item Uncertain merger opportunities. For instance a biotech may share data with a pharma who decides later on to merge with another firm or not to pursue the relationship. Or the regulator's decision is somewhat random.
      \item Endogenous merger choice by the regulator. Seems complicated but may be worth thinking about it as the paper should probably say something about guidelines for regulating data sharing.
    \end{enumerate}

\bibliographystyle{agsm}
\bibliography{biblio-synergies.bib}


\end{document}

