%  Synergies and mergers-Thursday, 19 November 2020
%
%  Created by Patrick on 2020-11-19.
%  Copyright (c) 2020 __Patrick Legros__. All rights reserved.
%
%!TEX encoding = UTF-8 Unicode  

\documentclass[a4paper]{article}
\usepackage[utf8]{inputenc}
\usepackage{amsthm,amsmath,amssymb,amsfonts}
\usepackage{url}
\usepackage{hyperref}
\usepackage[round]{natbib}
%\bibliographystyle{plainnat}
\usepackage[mathscr]{euscript}
\let\euscr\mathscr \let\mathscr\relax
\usepackage[scr]{rsfso}
\usepackage{graphicx}
\usepackage{float}
\usepackage{enumerate}
\usepackage{blindtext}
\usepackage{setspace}
\usepackage{eurosym}
\usepackage{hyperref}
\usepackage{pdflscape}
\usepackage{array, multirow}
\usepackage[dvipsnames]{xcolor}

\usepackage{todonotes}

\usepackage{tikz}
\hypersetup{
colorlinks,
citecolor=NavyBlue,
linkcolor=NavyBlue,
urlcolor=NavyBlue}
\usetikzlibrary{matrix}
\newtheorem{prop}{Proposition}
\newtheorem{lemma}{Lemma}
\newtheorem{corollary}{Corollary}
\newtheorem{ass}{Assumption}
\newtheorem{result}{Result}
\newtheorem{condition}{Condition}
\newtheorem{Theorem}{Theorem}
\newtheorem{Definition}{Definition}
\newtheorem{remark}{Remark}

\newenvironment{reusefigure}[2][htbp]
  {\addtocounter{figure}{-1}%
   \renewcommand{\theHfigure}{dupe-fig}% If you're using hyperref
   \renewcommand{\thefigure}{\ref{#2}}% Figure counter is \ref
   \renewcommand{\addcontentsline}[3]{}% Avoid placing figure in LoF
   \begin{figure}[#1]}
  {\end{figure}}
  
  \usepackage{titlesec}

%\setcounter{secnumdepth}{4}

\titleformat{\paragraph}
{\normalfont\normalsize\bfseries}{\theparagraph}{1em}{}
\titlespacing*{\paragraph}
{0pt}{3.25ex plus 1ex minus .2ex}{1.5ex plus .2ex}

 \usepackage{geometry}

 
 \newcommand{\E}{\mathbb E}
 \newcommand{\N}{\mathcal N}
\renewcommand{\th}{\hat\theta}
\renewcommand{\t}{\theta}
\renewcommand{\a}{\alpha}
\renewcommand{\b}{\beta}

\begin{document}


\title{Synergistic Information$^{3.0}$}
\author{Antoine Dubus and Patrick Legros\thanks{We would like to thank.}}
\date{\today}

%\author{Antoine Dubus\thanks{~~Université Libre de Bruxelles, ECARES; \href{mailto:antoine1dubus@gmail.com}{antoine1dubus@gmail.com}.}}

\maketitle

 
\textbf{Very preliminary, do not circulate.}

\baselineskip0.7cm


At time $0$, firm $2$ offers the option to firm $1$ to share amount of information $s^*$ at price $p$, in which case firm $2$ has full information about $\t$ when merger happens at time $1$. If firm $1$ does not share data, firm $2$ will offer a revelation mechanism $\{\alpha(\t),T(\t)\})$ to firm $1$.

At time $0$, let $\N$ be the set of types of firm $1$ that firm $2$ expects \emph{not to share}. We claim that $\N=[\t_0,\overline \t]$.
 
\paragraph{Optimal mechanism at $t=1$ given $\N$.} At this stage, firm $1$ has for outside option not to accept the mechanism and obtain a profit level of $\max_e eX -\frac{e^2}{2}=\frac{X^2}{2}$, independent of $\t$. The problem is a standard screening problem for firm $2$, with the caveat that firm $1$'s effort when choosing the element $(\alpha,T)$ solves $\max_e \alpha X(1+\t)+T-\frac{e^2}{2}$, or $e=X\frac{1+\t}{2}$, increasing in $\t$. 

If there is truth telling, type $\t$ has payoff
%
\[
U(\t)=\alpha(\t)^2 \frac{X^2(1+\t)^2}{2}+T(\t)
\]
%

Hence, by choosing $(\alpha(\th),T(\th)$, type $\t$ has payoff
%
\[
U(\th|\t)=\alpha(\th)^2 \frac{X^2(1+\t)^2}{2}+T(\th).
\]
%
Noting that $U(\th|\t)=U(\th)+\a(\th)X^2(\t-\th)\left(1+\frac{\t+\th}{2}\right)$, the incentive condditions $U(\t)\geq U(\th|\t)$ and $U(\th)\geq U(\t|\th)$ yield
%
\[
\a(\t) X^2(\t-\th)\left(1+\frac{\t+\th}{2}\right)   \leq U(\t)-U(\th) \leq \a(t)X^2(\t-\th)\left(1+\frac{\t+\th}{2}\right) 
\]
%
Therefore, as $\t>\th$, we must have $\a(\t)\geq \a(\th)$ and $U(\t)\geq U(\th)$.  The standard results follow: firm $2$ will want to bind the participation constraint of type $\t_0:=\inf \N$ and $U(\t)$ is an increasing function of $\t$.

Now, if a type shares information, firm $2$ extracts all surplus at the merger stage, and therefore type $\t$ has payoff from sharing equal to $\frac{(X-\t ls^*)^2}{2}+p$. Her outside option is the best of not doing anything and get $\frac{X^2}{2}$ or plays the mechanism $\{\a(\t),T(\t)\}$. 

This allows us to show that $\N$ is an interval $[\t_0,\overline t]$. Indeed, suppose that types $\t,\t'$ are in $\N$. Any type in $(\t,\t')$ has the option not ot share data and play the mechanims and obtain at least $U(\t|\th)$, which is strictly greater than $\max\left\{\frac{X^2}{2}, \frac{(X-\th ls^*)^2}{2}+p\right\}$: the first element is because $U(\t|\th)\geq U(\t_0|th)\geq \frac{X^2}{2}$ and the second element is because type $t$ prefers not ot share than to share. 

Incentive compatibility requires that for almost all $\t\in[\t_0,\overline \t]$, $\dot U(\t)=\a^2(\t) X^2(1+\t)$, hence in $2$'s problem we can replace $T(\t)$ by $-U(\t_0)-\int_{\t_0}^\t \a^(t)2 X^2(1+t)dt$, and after integration by parts, and binding $\t_0$'s participation constraint obtain the maximization problem
%
\begin{equation}
\begin{aligned}
    \frac{X^2}{1-F(\t_0)}\int_{\t_0}^{\overline \t}\a(\t) (1+\t)^2-\a^2(\t)(1+\t)\left(\frac{(1+\t)}{2}+\frac{1-F(\t)}{f(\t)}\right) dF(\t)\\
\end{aligned}
\end{equation}
This is maximized for 
%
\begin{equation}
    \a^*(\t)=\frac{(1+\t)}{(1+\t)+2 \frac{1-F(\t)}{f(\t)}}.
\end{equation}
%
Hence, conditional on firm $1$ not sharing, firm $2$'s expected payoff is 
%
\[
   \frac{X^2}{1-F(\t_0)}\int_{\t_0}^{\overline \t} \frac{(1+\t)^3}{2(1+\t)+\frac{1-F(\t)}{f(\t)}}f(\t)d\t
\]
%
\todo[inline]{Is it obvious that firm $2$ will always desire to merge? I think yes, but this should be established. Akin to showing that firm $2$ does not want to exclude some types.}



\subsection*{Equilibrium Data Sharing}
There are three possible cases: when $\t_0=0$ and there is no sharing of data, when $\t_0\in(0,\overline \t)$, and there is a mixture of data sharing by firms with low values of $\t$, and when $\t_0=\overline \t$ and all firms share data.

Because type $\t_0$ has no rent if firm $1$ does not share data (that is $U(\t_0)=\frac{X^2}{2}$) it must be the case that it has also zero rent if firm $1$ shares data. Hence, we must have $\frac{(X-\t_0 l s^*)^2}{2}+p=\frac{X^2}{2}$, or
%
\[
p=\frac{\t_0 l s^*}{2}(2X-\t_0 l s^*).
\]
%
It follows that when some types of firm $1$ decide to share data, firm $2$ has an expected payoff of 
%
\[
V(\t_0):=F(\t_0)\frac{\t_0 l s^*}{2}(2X-\t_0 l s^*)+X^2\int_{\t_0}^{\overline \t}\frac{(1+\t)^3}{2(1+\t)+\frac{1-F(\t)}{f(\t)}}f(\t)d\t
\]
%
The  higher $p$ is, the more likely that firm $1$ shares data, that is that $\t_0>0$. If the rents $\dot U(\t)$ that have to be given to firm $1$ ofr $\t>\t_0$ are large on average, firm $2$ may prefer to induce data sharing for all types, that is choose $\t_0=\overline \t$.

Somewhat unusual, if $\t_0\in(0,\overline \t)$, the equilibrium payoff of firm $1$ is not monotonic in $\t$: it is decreasing in $\t<\t_0$ because firms with high $\t$ lose more than lower types if there is data sharing and competition, hence are at a disadvantage when the merger happens.


TBC...


\section{Antoine's stuff}

If merger occurs:

\begin{itemize}
    \item Industry profits: $X(1+\t)e-\frac{e^2}{2}$, where e is an effort exerted by firm 1, inducing quadratic costs
    \item Firm 2 gets the whole profits, and gives a share $\a$ to firm 1 to exert positive efforts
    \item the profits of the firms are thus
    
    $$\Pi_2(\a(\t))= (1-\a(\t))X(1+\t)e-T(\t)$$

    \[\Pi_1(\a(\t))= \a(\t) X(1+\t)e-\frac{e^2}{2}+T(\t).\]
    
    \item firm 2 does not know the type of firm 1, and thus ignores whether the share $\a$ that maximizes its profits is enough to ensure participation.
    
    \item the optimal effort for firm 1 is $$e^*=\a(\t) X(1+\t)$$ 
    
    and $$\Pi_1^*=\frac{\a(\t)^2 X^2(1+\t)^2}{2}+T(\t)$$
    
    which is larger than $X$ for $$\t\geq \frac{\sqrt{2 (X-T(\t))}}{\a(\t) X}-1$$
    
    \item With a fixed $\a$ and a fixed $T$, the profits of firm 2 thus are
    
    $$\int_{\frac{\sqrt{2 (X-T)}}{\a X}-1}^{\overline \t}\a(1-\a)X^2\frac{(1+\t)^2}{2}-T dF(\t)$$
    
    \item firm 2 can offer a screening contract composed of a couple $(T(\t),\a(\t))$, such that the profit of firm 1 of type $\t$ selecting couple $(\a(\th), T(\th))$ is
    
    $$U(\hat \t|\t)=\a(\hat \t) X(1+\t)e+T(\hat \t)-\frac{e^2}{2}$$
    
    and with the equilibrium value of e
    
    $$U(\hat \t|\t)=\frac{\a^2(\hat \t) X^2(1+\t)^2}{2}+ T(\hat \t)$$
    
    \item incentive compatibility constraint implies  
    
    $$\frac{\a^2(\hat \t) X^2[(2+\t+\hat \t)(\t - \hat \t)]}{2}\leq U(\t)-U(\hat \t)\leq \frac{\a^2(\t) X^2[(2+\t+\hat \t)(\t - \hat \t)]}{2}$$
    \item This implies that $\a(\t)$ is non decreasing almost everywhere, and that 
    
    $$\dot U(\t)=\a^(\t)2 X^2(1+\t)$$
    
    \item The profit maximizing function of firm 2 is 
    
    \begin{equation}
        \begin{aligned}
       \max_{\a(.)}\int_{\underline \t}^{\overline \t}(1-\a(\t))\a(\t) X^2(1+\t)^2-T(\t)& dF(\t)&\\
       \iff \max_{\a(.)}\int_{\underline \t}^{\overline \t}(1-\a(\t))\a(\t) X^2(1+\t)^2-U(\t)&+\frac{\a^2(\t) X^2(1+\t)^2}{2} dF(\t)&\\
       U(\t)\geq& \frac{X^2}{2}&IR\\
       U(\t)\geq& U(\hat \t|\t)&IC
    \end{aligned}
    \end{equation}
    \item This is equivalent to maximizing

\begin{equation}
    \begin{aligned}
       &\int_{\underline \t}^{\overline \t}(1-\a(\t))\a(\t) X^2(1+\t)^2-U(\t)+\frac{\a^2(\t) X^2(1+\t)^2}{2} dF(\t)\\
       =&\int_{\underline \t}^{\overline \t}\a(\t) X^2(1+\t)^2-U(\t)-\frac{\a^2(\t) X^2(1+\t)^2}{2} dF(\t)\\
       =&X^2\int_{\underline \t}^{\overline \t}\a(\t) (1+\t)^2-\int_{\underline \t}^{\t}\a^2(\t)(1+\t)d\t-\frac{\a^2(\t) (1+\t)^2}{2} dF(\t)\\
       =&X^2\int_{\underline \t}^{\overline \t}\a(\t) (1+\t)^2-\a^2(\t)(1+\t)\left(\frac{(1+\t)}{2}+\frac{1-F(\t)}{f(\t)}\right) dF(\t)\\
\end{aligned}
\end{equation}
    \item This is maximized for 
    
    $$\a^*(\t)=\frac{(1+\t)f(\t)}{(1+\t)f(\t)+2-2F(\t)}$$
\end{itemize}




Question: how to model partial information sharing with moral hazard?

\begin{itemize}
    \item Without mergers, firm 1 makes profits $Xe-\frac{e^2}{2}$ 
    \item with information sharing, firm 1 makes profits $(X-ls^*)e-\frac{e^2}{2}$ and firm 2 makes profits $e\t s$
    \item sharing information thus decreases the incentives to effort of firm 1
    \item in turn, this changes the outside option of firm 1 when merging with prior information sharing:
    
    $$\a^2(\t) X^2(1+\t)\geq (X-ls^*)^2$$
    \item The optimal share of information is identical to above, however the transfer changes as the bargaining power of Firm 1 decreases with prior information sharing
\end{itemize}


\bibliographystyle{agsm}
\bibliography{biblio-synergies.bib}


\end{document}

