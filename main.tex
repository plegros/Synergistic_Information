\documentclass[a4paper]{article}
\usepackage[utf8]{inputenc}
\usepackage{amsmath}
\usepackage{amssymb}
\usepackage{url}
\usepackage{hyperref}
\usepackage[]{natbib}
%\bibliographystyle{plainnat}
\usepackage[mathscr]{euscript}
\let\euscr\mathscr \let\mathscr\relax
\usepackage[scr]{rsfso}
\usepackage{graphicx}
\usepackage{float}
\usepackage{enumitem}
\usepackage{blindtext}
\usepackage{setspace}
\usepackage{eurosym}
\usepackage{hyperref}
\usepackage{pdflscape}
\usepackage{array, multirow}
\newcommand*{\qed}{\hfill\ensuremath{\blacksquare}}%
\usepackage[dvipsnames]{xcolor}
\usepackage{tikz}

\usepackage{todonotes}

\hypersetup{
colorlinks,
citecolor=NavyBlue,
linkcolor=NavyBlue,
urlcolor=NavyBlue}
\usetikzlibrary{matrix}
\newtheorem{prop}{Proposition}
\newtheorem{lemma}{Lemma}
\newtheorem{corollary}{Corollary}
\newtheorem{ass}{Assumption}
\newtheorem{result}{Result}
\newtheorem{condition}{Condition}
\newtheorem{Theorem}{Theorem}
\newtheorem{Definition}{Definition}

\newenvironment{reusefigure}[2][htbp]
  {\addtocounter{figure}{-1}%
   \renewcommand{\theHfigure}{dupe-fig}% If you're using hyperref
   \renewcommand{\thefigure}{\ref{#2}}% Figure counter is \ref
   \renewcommand{\addcontentsline}[3]{}% Avoid placing figure in LoF
   \begin{figure}[#1]}
  {\end{figure}}
  
  \usepackage{titlesec}

%\setcounter{secnumdepth}{4}

\titleformat{\paragraph}
{\normalfont\normalsize\bfseries}{\theparagraph}{1em}{}
\titlespacing*{\paragraph}
{0pt}{3.25ex plus 1ex minus .2ex}{1.5ex plus .2ex}

  \usepackage{geometry}
 \geometry{
 a4paper,
 total={170mm,257mm},
 left=2cm,
 right=2cm,
 top=1.2cm,
 bottom=1.2cm,
 }
 
 \newcommand{\E}{\mathbb E}
\renewcommand{\th}{\hat\theta}
\renewcommand{\t}{\theta}
\newcommand{\s}{\sigma}
\begin{document}


\title{Synergistic information\thanks{We would like to thank.}}

\date{\today}

%\author{Antoine Dubus\thanks{~~Université Libre de Bruxelles, ECARES; \href{mailto:antoine1dubus@gmail.com}{antoine1dubus@gmail.com}.}}

\maketitle

\begin{abstract}

\noindent 

\end{abstract}
 
\textbf{Very preliminary, do not circulate.}

\baselineskip0.7cm


\section{Model}


\begin{itemize}
	\item Two firms, 1 and 2. 
	\item Firm $1$ is endowed with a data set $X$. Firms make profits using data
	\item Firm 2 wants to merge with Firm 1. The merged entity makes profits $\Pi_m=X(1+\t)$ where $\t$ is the synergistic value of information owned by Firm 1 when combined with the data owned by firm $2$. 
	\item Firm 2 and a regulator believe that the type of Firm 1 is uniformly distributed over $[\underline{\t},\overline{\t}]$.
    \item Firm $1$ can share $s$ information from the total data set $X$, which is verifiable by Firm $2$ and the regulator.
	\item The regulator will choose whether to allow the merger or not. The regulator tradeoffs the cost from increased market power (assumed to simplify to be proportional to the industry profit) and the gain from synergies. There is some uncertainty on the weight $w$ the regulator will put on synergies, and this uncertainty is resolved only at the time the regulator evaluates the merger proposal, that is after firms $1,2$ have played their mechanism, and $s$ has been shared.
	\item If the merger fails, the information that has been shared by Firm $1$ can be used by Firm 2 during the competition, and information sharing changes the profit of Firm $1$: $\Pi_{c1}=X-\t l(s)$; and of Firm 2: $\Pi_{c2}=\t s$.\footnote{\href{
https://www.justice.gov/atr/case-document/competitive-impact-statement-108}{Competitive Impact Statement, U.S. Department of Justice, March 19, 2003.}} We assume Inada conditions $l(0)=0,l'(0)=0$.
	\item There are two cases to consider:
\begin{itemize}
    \item $l(s)\geq 0$: sharing information increases competition and incurs a loss for a firm
    \item $l(s)< 0$: sharing information facilitates coordination and increases the profits of a firm (à la \cite{vives1984duopoly} for instance)
 \end{itemize}
    \item Firm 2 offers a menu $(s,T(s))$ and firm $1$ chooses -- as a function of her type $\theta$ -- how much to share. Conditional on observing $s$, both firms decide to approach the regulator to authorize the merger. We assume that there is no renegotiation on $T(s)$ at this stage.
    \item If the merger occurs, Firm $1$ agrees to make a transfer  $T(s)$ to Firm 2 and be residual claimant on the profit.\footnote{%
    This payment scheme is without loss of generality. What matters is that the profit of firm $1$ in case of merger depends on its true type, for otherwise revelation of information is impossible. Also, for incentive compatibility, it is clear that if two $\theta$s share the same $s$, they will claim to be of the type that minimizes $T(\theta,s)$.}
    
\end{itemize}

\paragraph{The timing of the game is the following.} ($U[a,b]$ denotes the uniform distribution on $[a,b]$; we could use more general distributions...)
\begin{enumerate}\setlength\itemsep{0em}
    \item Firm $2$ offers a menu $(s,T(s)$, $s$ being the sharing of data that firm $1$ should do and $T(s)$ is the price firm $2$ agrees to pay to firm $1$ if the merger is agreed and the regulator authorizes it. \todo[inline]{Or firm $2$ makes a take-it-or leave-it offer to firm 1 \textbf{after} $s$ is sunk... may be a useful variation, but quid of renegotiation, etc? Perhaps better to keep commitment to $T(s)$. But explain why a big firm 2 will stick to the terms (not try to hold up firm $1$ after $s$ is sunk) and refuse any offer from firm 1 of a lower price.}
    \item Firm 1 privately learns its type $\t\sim U[\underline{\t},\overline{\t}]$ and plays $s$.
    \item If both firms anticipate to be better off via a merger, they approach the regulator and ask for the merger to be approved. At this point, the regulator sees $s,T(s)$.
    \item The regulator observes a draw $w$ from a distribution $F$ on $\mathbb R_+$ with a continuous density. The regulator decides to allow or to prevent the merger. The market structure, profits and welfare, are realized.
\end{enumerate}
%
\subsection{The Regulator's Problem}
   The regulator maximizes a social welfare function that weights the social cost of high industry profits and the social benefit of synergies. Contrary to the usual view that synergies are created only during the merger, synergies endogenously happen without a merger if firm $1$ shares some of its data with firm $2$. Hence, when evaluating a merger proposal, the regulator will compare the \emph{relative synergy gain} to the \emph{relative industry profit gain.}

   At the time the regulator has to evaluate a merger, firm $1$ has already shared $s$ with firm $2$, following a strategy $\sigma(\t)$ in the sharing game.  Let $\sigma^{-1}=\{\t|\sigma(\t)=s\}$ be the set of types of firm $1$ consistent with a sharing of $s$ given the equilibrium strategy $\sigma$. Hence, conditional on observing $s$, the regulator believes that the expected synergy is the conditional expectation $$\t_s:=\E[\t|\t\in\sigma^{-1}(s)].$$
     
    If the merger is allowed and firm $2$ takes control of firm $1$, the loss from industry profit is $-[X(1+\t_s)]$ while the synergy gain is $w[X(1+\t_s)]$, hence welfare is
    
    $$W_m=(w-1)X(1+\t_s)$$
    
    If the merger is prevented or the firms decide not to go ahead with it, firm $2$ benefits from the information given by firm $1$ (by a factor of $\t s$, where $\t$ is the true state of the world), and firm $1$ has a loss of $\t_s l(s)$, implying that the total industry profit increases by $\t s -\t l(s)$. Therefore, the regulator believes that the realized industry profit will be $X_1-\t_s l(s)+X_2+\t_s s$ if he does not authorize the merger, while the synergy benefit would be $w[X+\t_s s]$.\footnote{%
    As it should be, the regulator ignores the market loss $\t_s l(s)$ to firm $1$.} Hence the expected welfare under competition is
      $$W_c=-X+\t_s l(s)-\t_s s + w (X+\t_s s)$$
    The regulator allows the merger when $W_m\geq W_c$, or
    
    \begin{equation}
           w\geq w^*(s):=1 + \frac{l(s)}{(X-s)}
    \end{equation}
%
Our specification implies therefore that the probability that the merger is authorized depends only on the share $s$ but not on the beliefs of the regulator about $\t$.

At the ex-ante stage, when firms play their mechanism to share information, the probability that a merger will be approved when firm $1$ shares $s$ with firm $2$ is then equal to  
\[
a(s):=1-F(w^*(s)).
\]


\section{Playing the Regulatory Game}

At the stages where firm $2$ offers a menu $(s,T(s)$ and firm 1 plays $s$, the probabilities $a(s)$ are given because if called upon to act, the regulator allows the merger only depending on the value of $s$. As is standard, there are two types of incentive conditions to be satisfied: one related to using the requested share, and the other related to telling the truth.

\todo[inline]{Patrick a ce stade je ne suis pas sur de s'il faut les deux contraintes, si je comprends bien la seule contrainte nécessaire est celle de using the requested share, qui revient à choisir quelle paire du menu prendre. Le truthtelling vient naturellement de la séparation qui en suit}

If these conditions are satisfied, it must also be the case that firm $1$ prefers to play the mechanism than not playing it, and foregoing the possibility of a merger. 

The interim individual rationality constraint of firm $1$ is then
\begin{equation}\label{cond:IR}
    \begin{aligned}
    &a(\s(\t))(X (1+\t)- T(\t,\s(\t))+(1-a(\s(\t)))(X-\t l(\s(\t)))\geq X  \\
    \implies &a(\s(\t))(X \t- T(\t,\s(\t))+\t l(s(\t)))\geq \t l(\s(\t))
    \end{aligned}
\end{equation}

\paragraph{The Incentives to Share.}

%Suppose first that firm $1$ of type $\t$ selects the pair $(\s(\t),T(\s(\t))$ from the menu proposed by firm 2. As firm $1$ shares $s=\s(\t)$, there will be a merger with probability $a(\s(\t))$, hence firm 1's expected profit is
%\[
%U(\t):= a(s(\t))(X(1+\t)-T(\t))+(1-a(s(\t)))(X_1-\t l(s(\t))),
%\]
%while by deviating to $s\neq \s(\t)$ there is competition for sure and firm $1$'s payoff is $X-\t l(s)$. Clearly, the optimal deviation is $s=0$. Hence,  following the recommended share while telling the truth is optimal when the IR constraint \eqref{cond:IR} is satisfied.



If type $\t$ chooses $s=\s(\th)$ when the IR condition for type $\th$ is satisfied since by deviating to $\th$ and doing $s(\th)$, type $\t$ gets

  \begin{align*}
    U(\th|\t)&:=a(s(\th))(X(1+\t)-T(\th))+(1-a(s(\th)))(X-\th l(s(\th)))
  \end{align*}
%which is greater than $X$ when the IR constraint of $\th$ is satisfied.

It remains therefore to verify that type $\t$ does not want to deviate to $\th$ by sharing $\s(\th)$, that is that $U(\s(\t))\geq U(\s(\th)|\t)$. Now,

\begin{align*}
U(\s(\th)|\t)=U(\th)+a(\s(\th))X(\t-\th)
\end{align*}

and therefore the two incentive constraints $U(\s(\th)|\t)\leq U(\t)$ and $U(\s(\t)|\th)\geq U(\th)$ yield
\[
a(\s(\th))X(\t-\th)\leq U(\t)-U(\th)\leq a(\s(\t))X(\t-\th)
\]
Therefore, $U(\t)$ is a monotonic non-decreasing function.

Consider now the variations of $a(.)$ and of $\s(\t)$:

\[
a(\s(\th))\leq a(\s(\t))
\]

and 

\[
a(\s(\t))=1-F(w^*(\s(\t)))
\]

The variations of $a(\s(\t))$ are therefore opposite to those of $\frac{l(\s(\t)}{X-\s(\t)}$

$\frac{\partial}{\partial \s}\frac{l(\s(\t)}{X-\s(\t)}=\frac{l'(\s(\t))(X-\s(\t))+l(\s(\t))}{(X-\s(\t))^2}$

When this first degree derivative is positive, $\frac{l(\s(\t)}{X-\s(\t)}$ increases with $\s(.)$, $a(.)$ decreases with $\s(.)$, and thus, as $a(\s(\th))\leq a(\s(\t))$, $\s$ decreases with $\t$.

On the opposite, when the first degree derivative is negative, $\s$ increases with $\t$.

\subsection{Conditions for pooling to occur}


Assume that there exist $\t_0,\tilde{\t}$ such that $\forall \t\in[\t_0, \tilde{\t}], \s(\t)=\s(\tilde{\t})=\tilde{s}$. 

The utilities of types $\t,\th \leq \tilde{\t}$ are:

  \begin{align*}
    U(\t)&:=a(\tilde{s})(X(1+\t)-T(\tilde{s}))+(1-a(\tilde{s}))(X-\t l(\tilde{s}))\\
    U(\th)&:=a(\tilde{s})(X(1+\th)-T(\tilde{s}))+(1-a(\tilde{s}))(X-\th l(\tilde{s}))
  \end{align*}
  
By continuity around $\tilde{s}$, deviation is not profitable if:

  \begin{align*}
    &\left.\frac{\partial (a(s)(X(1+\t)-T(s))+(1-a(s))(X-\t l(s)))}{\partial s}\right|_{\tilde{s}}=0\\
   &\left.\frac{\partial (a(s)(X(1+\th)-T(s))+(1-a(s))(X-\th l(s)))}{\partial s}\right|_{\tilde{s}}=0\\
   \implies&\t (a'(\tilde{s})(X+l(\tilde{s}))-l'(\tilde{s})(1-a(\tilde{s})))=\th (a'(\tilde{s})(X+l(\tilde{s}))-l'(\tilde{s})(1-a(\tilde{s})))=a'(\tilde{s})T(\tilde{s})+a(\tilde{s})T'(\tilde{s})=0
  \end{align*}

As this local condition does not depend on the value of $\t$, it is verified for all types, even outside of $[\t_0, \tilde{\t}]$.

  
Globally, the incentive compatibility constraint must be verified for all types in $[\t_0, \tilde{\t}]$, $\forall s$:

  \begin{align*}
    a(\tilde{s})(X(1+\t)-T(\tilde{s}))+(1-a(\tilde{s}))(X-\t l(\tilde{s}))\geq a(s)(X(1+\t)-T(s))+(1-a(s))(X-\t l(s))
  \end{align*}


Whether this condition is satisfied depends on the the value of the transfer $T$, chosen by Firm 2.

Firm 2 can define $T$ such that pooling or separation occurs. The value of $T$ also characterizes the set over which pooling occurs. We need to compare profits in all cases to see what is optimal for Firm 2.

\todo[inline]{Patrick dans anton et yao dans ce cas ils se concentrent sur l'equilibre separating et laissent le pooling de coté}





\section{Extensions}


\subsection{Anticompetitive effects of information sharing}

The anticompetitive effects of information sharing is captured in our model by the cost of sharing for firm 1 $\tl(\s(\t))$, which can be positive (procompetitive) or negative (anticompetitive)


\subsection{Negative social value of data}

In this case $w$ is negative.

A third case could be that $w\in[\underline{w},\overline{w}]$ with $\underline{w}\leq 0$ and $\overline{w}\geq 0$. In this case it is not clear whether data is beneficial or detrimental. This could corresponds for instance to a tradeoff between costs and benefits of data. The choice of the regulator could be driven by and exogenous doctrine for instance.



\subsection{Bilateral signaling}



\bibliographystyle{plain}
\bibliography{Bibliography}





\end{document}

