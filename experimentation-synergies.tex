%  Synergies and mergers-Thursday, 19 November 2020
%
%  Created by Patrick on 2020-11-19.
%  Copyright (c) 2020 __Patrick Legros__. All rights reserved.
%
%!TEX encoding = UTF-8 Unicode  

\documentclass[a4paper,leqno]{article}%leqno to number equations on the left
\usepackage[utf8]{inputenc}
\usepackage{amsthm,amsmath,amssymb,amsfonts}
\usepackage{url}
\usepackage{hyperref}
\usepackage[round]{natbib}
%\bibliographystyle{plainnat}
\usepackage[mathscr]{euscript}
\let\euscr\mathscr \let\mathscr\relax
\usepackage[scr]{rsfso}
\usepackage{graphicx}
\usepackage{float}
\usepackage{enumerate}
\usepackage{blindtext}
\usepackage{setspace}
\usepackage{eurosym}
\usepackage{hyperref}
\usepackage{pdflscape}
\usepackage{array, multirow}
\usepackage[dvipsnames]{xcolor}

\usepackage{todonotes}

\usepackage{tikz}
\hypersetup{
colorlinks,
citecolor=NavyBlue,
linkcolor=NavyBlue,
urlcolor=NavyBlue}
\usetikzlibrary{matrix}
\newtheorem{prop}{Proposition}
\newtheorem{lemma}{Lemma}
\newtheorem{corollary}{Corollary}
\newtheorem{ass}{Assumption}
\newtheorem{result}{Result}
\newtheorem{condition}{Condition}
\newtheorem{Theorem}{Theorem}
\newtheorem{Definition}{Definition}
\newtheorem{remark}{Remark}
\newtheorem{example}{Example}

\newenvironment{reusefigure}[2][htbp]
  {\addtocounter{figure}{-1}%
   \renewcommand{\theHfigure}{dupe-fig}% If you're using hyperref
   \renewcommand{\thefigure}{\ref{#2}}% Figure counter is \ref
   \renewcommand{\addcontentsline}[3]{}% Avoid placing figure in LoF
   \begin{figure}[#1]}
  {\end{figure}}
  
  \usepackage{titlesec}

%\setcounter{secnumdepth}{4}

\titleformat{\paragraph}
{\normalfont\normalsize\bfseries}{\theparagraph}{1em}{}
\titlespacing*{\paragraph}
{0pt}{3.25ex plus 1ex minus .2ex}{1.5ex plus .2ex}

 \usepackage{geometry}

 
 \newcommand{\E}{\mathbb E}
 \newcommand{\N}{\mathcal N}
\renewcommand{\th}{\hat\theta}
\renewcommand{\t}{\theta}
\renewcommand{\a}{\alpha}
\renewcommand{\b}{\beta}
\newcommand{\s}{\sigma}

\begin{document}


\title{Synergistic Information$^{3.0}$}
\author{Antoine Dubus and Patrick Legros\thanks{We would like to thank.}}
\date{\today}

%\author{Antoine Dubus\thanks{~~Université Libre de Bruxelles, ECARES; \href{mailto:antoine1dubus@gmail.com}{antoine1dubus@gmail.com}.}}

\maketitle

 
\textbf{Very preliminary, do not circulate.}

\baselineskip0.7cm
Mergers fail, claimed efficiencies at the time of merger review are often not realized, or is there are realized, they are not passed through to consumers. Is this because firms pretend the synergies, efficiency gains are present in the hope of convincing authorities to give them an instrument for more market power? Is it because the firms haven't done the homework and incorrectly evaluated the extent of synergies? Or is it because synergies exist ``on average'' only?

If firms collaborate before engaging into a full merger review, they may learn the extent of potential synergies. Joint ventures, sharing of information are typical cases of such collaborations. You know algorithms and data are the source of value created to customers, the sharing of data among firms is another example of such collaboration.

Arrow famously pointed out the difficulty of inducing such collaborations among competitors, especially if what is shared is an idea that can be replicated at no cost. But even if there is a cost to replication, providing assets to competitors enhances their ability to compete, hence is costly for the firm that shares these assets. Our point in this paper is to show that this competitive disadvantage can turn into more efficient merger decisions, which is at the benefit of the firms, and sometimes consumers.



\section{Literature}
\begin{itemize}\setlength\itemsep{-1em}
    \item Competition policy for the digital era
    
    \cite{tirole2020competition}; \cite{scott2019committee}; \cite{cremer2019competition}; \cite{cabral2020merger}
    
    None of them has considered data as a motivation to merge. We fill a gap in this literature.
    
    \item Optimal merger policy when firms have private information \cite{Besanko1993}
    \item Information sharing in oligopolies
    
    \cite{vives1984duopoly, gal1986information} sharing information can have pro or anti competitive effects depending on the nature of competition (cournot vs bertrand)
    
    They justify that we look at $l$ positive and negative
    \item Disclosure to consumers 
    \item Data as assets
    
    \cite{stucke2016introduction} discuss how mergers are now motivated by the acquisition of a firm data set
    \item Computer science and data synergies:
    
    \cite{bertschinger2014quantifying, Griffith2014, olbrich2015information} discuss how information synergies can arise when merging two data sets, 
    
    justify why we focus on information synergies
    \item Joint ventures before merger
    \item incomplete contracts and cooperative investments \citep{Che1999}
\end{itemize}



\section{Concrete example}

\begin{itemize}
    \item pre merger information sharing
    
    \href{https://sites-herbertsmithfreehills.vuturevx.com/46/12874/compose-email/the-altice-case--a-costly-warning-not-to-engage-in-gun-jumping-before-receiving-merger-control-clearance.asp}{Altice/OTL} The FCA found that Altice and SFR engaged in an extensive exchange of commercially sensitive information (including individualised trade data and future forecasts)
    
    \href{https://www.twobirds.com/en/news/articles/2020/global/double-caution-gun-jumping-risks-in-m-and-a-transactions}{Gun-jumping examples}
    \item Acquisition failures:
    
    \href{https://www.investopedia.com/articles/insights/061816/4-cases-when-ma-strategy-failed-acquirer-ebay-bac.asp}{ebay/skype}: lack of complementarity
    
    \href{https://salessynergy.net/the-biggest-acquisition-disasters-that-put-companies-into-quite-a-bit-of-trouble/}{google/motorola}: android bugs on device
    
    \href{https://www.investopedia.com/articles/financial-theory/08/merger-acquisition-disasters.asp#:~:text=The\%20consolidation\%20of\%20AOL\%20Time,combination\%20up\%20until\%20that\%20time}{sprint/nextel} lack of common culture
    
    \item Regulator forbids merger:
    
    \href{https://www.livemint.com/companies/news/aurobindo-pharma-calls-off-1-billion-deal-with-sandoz-after-failing-to-get-ftc-nod-11585801128011.html}{Aurobindo/sandoz}
    
    \item Profit sharing mechanisms between two firms: \href{https://www.nytimes.com/2020/11/21/us/politics/coronavirus-vaccine.html?referringSource=articleShare}{Politics, Science and the Remarkable Race for a Coronavirus Vaccine}.
\end{itemize}





\section{Model}
\begin{itemize}
    \item There are two firms, indexed by $1,2$, where $2$ is a dominant firm (Google) and $1$ is a firm that has developed a new product of platform and has a stock of data of mass $1$ generated by this activity. Firm $1$ is able to give a utility level of $u$ to its customers and absent other competition can fix a price equal to $u$, making a profit of $u$ (assume a mass one of customers interested by the product.)
    \item Firm $2$ has developed other products and has its own stock of data. Combining data from $1$ and $2$ will enhance the value to customers to $v(\t)$, where $v(0)=0$ and $v(\t)$ increasing in $\t$. (That $v(0)=u$ reflects the fact that absent synergies the value to customer is the same whether firm $1$ offers the product or the merged firm $[12]$ offers the product.)
    \item The value of $\t$ is unknown to firms, but each know that it has a distribution $F(\t)$, with continuous density and no atom.
    \item Generating synergies is costly, it requires for instance the development of new algorithms or code, further marketing efforts. Let $c$ be this cost. 
    \item [No sharing]
    \item Firm $1$ can share $s\%$ of its data with firm $2$, possibly at an agreed upon price $T(s)$: at this time of sharing, firms $1,2$ only know that $\t$ has distribution $F(\t)$. Upon receiving $s$, firm $2$ can 
    \begin{itemize}
        \item either invest $c$, in which case the synergy $\t$ is learned (to simplify by both firms, the case where firm $2$ gets this information privately is for an extension), and it is known that the product provides value $v(\t)s$ to customers if a share $s$ of the data of firm $1$ is used. Once $\t$ is known, Firm 2 can then make a TIOLI offer to firm $1$ for creating a merger, or can choose to use the information to compete with firm $1$. If there is a merger customers will have value $v(\t)$ (since all data from firm $1$ is part of the assets of the merged firm). If there is not a merger, firm $2$ has a product competing with that of firm $1$ that provides value $v(\t)s$ to customers.
        \item Or not invest. In this case, firms $1,2$ still do not know the extend of the synergies and decide for a merger under imperfect information.
    \end{itemize}
    \item If firm $1$ does not share data, merger happens under imperfect information. (Note that sharing $s=0$ is equivalent to not sharing because even if firm $2$ invests $c$ the value to customers is $v(\t)s\equiv 0$ independently of the true value of $\t$, hence firm $2$ cannot compete with firm $1$.)
\end{itemize}


\section{Analysis}
\subsection{Competition}
Suppose that firm $1$ shares $s>0$, and let us ignore for the moment the possibility of a merger. If firm $2$ invests, it can provide its customes a value $u(\t)s$ while firm $1$ can provide a value $u$. Assuming Bertrand competition, it follows that the equilibrium price paid by the consumers and the profit per consumer are
\begin{align}\label{comp}
\begin{cases}
    p=v(\t)s-u,\; \pi_1(\t,s)=0,\; \pi_2(\t,s)=v(\t)s-u & \text{ if }v(\t)s-u\geq 0\\ 
    p=u-v(\t)s\; \pi_1(\t,s)=u-v(\t)s,\; \pi_2(\t,s)=0 & \text{ if }v(\t)s-u\leq 0.
\end{cases}
\end{align}
If firm $2$ does not invest, its profit per consumer is equal to zero, that of firm $1$ is equal to $u$ and synergies are not learned.

If mergers are not possible, it should be clear that firm $1$ has no incentive to share information. Note that absent a merger, firm $2$ will invest in $c$ only if 
%
\[
\int_{\t:v(\t)s-u\geq 0}(v(\t)s-u) dF(\t)\geq c.
\]
The complication is that investing in $c$ may provide useful information for negotiating the purchase of assets of firm $1$ at the merger phase, hence firm $2$ may decide to invest in $c$ even if the previous inequality fails.

\subsection{Merger}
%
Suppose that no data is shared. At the time of the merger the expected value is equal to $u$ and the merged firm has access to the full stock of data $s=1$. Therefore, the expected value if there is no investment is equal to $u$ and is equal to $\int v(\t)dF(\t)-c$ if there is investment. The value of the merger is therefore
%
\[
W^M:=\max\left\{u,\E[v(\t)]-c.\right\}
\]
%
Note that if there is investment, the merged firm over-invest when $v(\t)-c<u$.

Suppose that the firms agree that firm $1$ will share $s$ with firm $2$, and that firm $2$ agrees to pay $T(s)$ to firm $1$ for this amount of data. 

Upon receiving $s$, firm $2$ can decide to invest $c$ in order to learn $\t$. In this case, the two firms anticipate payoffs $\pi_i(\t,s)$ as given by \eqref{comp} if there is no merger. Because, $W^M\geq \pi_1(\t,s)+\pi_2(\t,s)$, a merger is always beneficial. Firm $2$ can make at TIOLI offer to buy firm $1$'s asset at a price $p(\t,s)$ that will make firm $1$ indifferent between merging and not merging, that is 
%
\begin{equation}\label{merger-price}
    p(\t,s):=\pi_1(\t,s).  
\end{equation}
%
It will be useful to use the notation
%
\[
\s:=v^{-1}
\]
clearly, $\s(s)$ is an increasing function of $s$ and $v(\s(s))=s$. Using \eqref{comp}-\eqref{merger-price}, if $\t\geq \s(\frac{u}{s})$, firm $1$ has a zero profit if there is competition, hence firm $2$ can merge with firm $1$ by offering a zero price and get the full surplus $v(\t)$. If $\t< \s(s)$, firm $1$ makes a profit if there is competition and will merge only if the price is at least equal to $u-v(\t)s$; hence firm $2$ can make a profit of at most $(1+s)v(\t)-u$ from the merger, which is greater than her payoff under competition only if $\t\geq \s(\frac{u}{1+s})$. It follows that the expected payoff of firm $2$ of paying $T(s)$ to get $s$ and investing $c$ following sharing of data is 
%
\begin{equation}\label{value-merger-firm2}
    \int_{\s(u/(1+s))}^{\s(u/s)} ((1+s)v(\t)-u)dF(\t)+\int_{\s(u/s)}^\infty v(\t)dF(\t)-c-T(s)
\end{equation}
%
By contrast firm $1$ has an expected payoff of 
\begin{equation}\label{value-merger-firm1}
    \int_{0}^{\s(u/s)}(u-v(\t)s)dF(\t)+T(s).
\end{equation}
%
Therefore, it must be the case that firm $2$ offers a price
%
\[
T(s):=u-\int_{0}^{\s(u/s)}(u-v(\t)s)dF(\t),
\]
%
for sharing $s$ and firm 2 has an expected payoff (substituting $T(s)$ in \eqref{value-merger-firm2}) equal to 
%
\[
w(s)=\int_0^{\s(u/(1+s))}(u-v(\t)s)dF(\t)+\int_{\s(u/(1+s))}^\infty v(\t)dF(\t)-u-c.
\]
%
The variation of this value is
%
\[
w'(s)= -\int_0^{\s(u/(1+s))}v(\t)dF(\t).
\]
%
\begin{example}
For instance, if $s\in[0,1]$ and $\t$ is uniformly distributed on $[0,1]$ and $v(\t)=\t$,
%
\[
w'(s)=u^2 \left(\frac{1}{2s^2}-\frac{1}{(1+s)^2}-\frac{1}{(1+s)^3}\right)
\]
%
which is positive for all $s\in[0,1]$. Hence, in this example firm $2$ will ask firm $1$ to share all its data.    
\end{example}

The alternative is not to share data. In this case, firm $2$ makes a TIOLI offer to buy firm $1$ at price $u$ and firm $2$ makes profit $w(0):=W^M-u$. 

Note that $\lim_{s\downarrow 0}w(s)=\int_{\s(u)}^\infty (v(\t)-2u)dF(\t)-c$ and \emph{is not equal } to what happens if there is no sharing: firm $2$ gets then $W^M-u=\max\{0, \E[v(\t)]-u-c\}$. It remains therefore to show that the optimal sharing dominates no sharing for firm $2$. We continue here with the specification in the example. Then $\E[v(\t)]=\int_0^1 \t d\t=\frac{1}{2}$, and firm $2$' payoff if there is no sharing is 
%
\[
w_0:=\max(0,\frac{1}{2}-u-c).
\]
%
while under sharing it is 
%
\[
w(1) = \frac{1}{2}+\frac{u^2}{8}-u-c. 
\]
%

\paragraph{Suppose $u+c\geq \frac{1}{2}$,} that is $w_0=0$. It is not beneficial to invest in learning the synergies if there is no sharing and a merge. In this case, the merger is neutral for consumers and the firms, and sharing is optimal whenever  and 
%
\begin{itemize}
    \item $c<\frac{1}{2}$ and $u\in \left[\frac{1}{2} (1-2 c),4-2 \sqrt{2 c+3}\right]\cup [2 \sqrt{2c+3}+4,\infty]$.
    \item $c=\frac{1}{2}$ and $u=0$ or $u\geq 8 $
    \item $c>\frac{1}{2}$ and $u\geq 2\sqrt{2 c+3}+4$
\end{itemize}
%


\paragraph{Suppose now $u+c<\frac{1}{2}$,} and $w_0=\frac{1}{2}-(u+c)$. Then sharing is optimal when (necesarily $c<\frac{1}{2}$)
%
\[
    0\leq u<\frac{1}{2} (1-2 c)
\]


\bibliographystyle{agsm}
\bibliography{biblio-synergies.bib}


\end{document}

